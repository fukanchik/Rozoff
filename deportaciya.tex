****** Депортация. мини-роман - трансутопия ******
Rozoff
Депортация. (мини-роман - трансутопия)

Пролог. CNN, Лантон, остров Тинтунг, округ Нельсон, Меганезия. Камера. Эфир.

В кадре – широко улыбающийся солидный мужчина, на фоне площади, окруженной густым фигурно подстриженным цветущим кустарником. Посреди площади – серебристое изваяние юной девушки, одетой в лава-лава, на угловатом каменном постаменте.

- Итак, Меганезия - в центре громкого международного скандала, а я – на центральной площади ее столицы, Лантона, расположенного на острове Тинтунг. Раньше здесь была резиденция губернатора, но во время так называемой алюминиевой революции, ее взорвали аммоналом. Остался только вот этот кусок цоколя, на котором сейчас установлен памятник королеве Лаонируа, или, как называют ее местные жители queen Lao. Памятник, кстати отлит все из того же алюминия.

Королева Лаонируа - это псевдоним, а настоящее имя - Лайза Корн. Уроженка Бостона, дочь афроамериканца и китаянки, мисс Корн в начале своей карьеры играла в мюзиклах виртуального театра Николаса Скиннера. Когда Скиннер был обвинен в уклонении от налогов, они оба покинули США и перебрались в Лантон, в то время – административный центр британской Океании. Здесь они ввязались в авантюру батакских националистов, мечтавших восстановить монархию, существовавшую до британского доминиона. Они выдавали мисс Корн за наследницу древней королевской фамилии, используя ее внешнее сходство с аборигенкой. Эта топорная афера осталась бы лишь в анекдотах, если бы во время стычки батаков с колониальными властями мисс Корн не была случайно застрелена полицейским. В этот момент она исполняла песню «Go down, Moses» Луи Армстронга.

Мертвая мисс Корн оказалась гораздо убедительнее в роли королевы Лаонируа, чем живая, а слова: «Let my people go» - припев из песни и строка из библейской книги Exodus - стали символом всех местных ультра. На следующий день против толпы, скандирующей слова «Отпусти мой народ», обращенные некогда Мозесом к фараону, были применены водометы и слезоточивый газ. В ответ ультра пригласили наемников – хуту и военных инструкторов из Вьетнама, которые устроили в Лантоне и на всем острове Тинтунг минную войну. Всего за сутки оказались разрушены административные здания и казармы колониальных войск. Рейды наемников вынудили британский контингент покинуть сначала остров, затем весь архипелаг Нельсона, а затем и близлежащие архипелаги. Восставшие учредили независимую конфедерацию Меганезия из четырех архипелагов и приняли «Великую Хартию» - странную смесь коммунизма, фашизма и руссоизма.

Самопровозглашенный национальный конвент назначил техническое правительство и учредил избираемый по жребию верховный суд с драконовскими полномочиями. Из отборных наемников был сколочен полицейский корпус, исполнявший решения этого суда. По архипелагу прокатилась волна репрессий и национализаций. Партия батакских националистов попыталась напомнить о своей роли в захвате власти – но ее выступление было жесточайшим образом потоплено в крови. Революция, как всегда, пожирала своих детей. Верховный суд запретил вообще все политические партии и государственные институты, объявив государство антинародной идеей и оплотом старого режима.

Эти реформы привлекли на архипелаги значительное число левацких группировок из Южной Америки. Из них тут же были образованны вооруженные силы. Свое неумение воевать они компенсировали крайней жестокостью при совершении террористических актов. После кровавого инцидента с американо-японской концессией на острове Панджонг, Верховный суд объявил терроризм официальной военной доктриной. Это вызвало полугодичную международную изоляцию Меганезии, которая была прервана только по причине необходимости поддержания судоходства в этом регионе. К тому времени к конфедерации присоединилось еще несколько архипелагов, из-за чего многие тихоокеанские маршруты оказались как бы во внутренних водах Меганезии, и, во всяком случае, в двухсотмильной зоне этой страны. Скоро этот экономический регион освоили частные инвесторы, привлеченные низкими налогами. В Меганезии, в отличие от коммунистических стран, свобода частного бизнеса в основном сохранена, а кое-где даже шире, чем на Западе. Природные ресурсы и ряд отраслей экономики национализированы, а практика так называемых социальных наблюдателей выглядит порой просто жутко, но это не останавливает рисковых бизнесменов, привлеченных налоговым пряником.

При всей абсурдности образовавшегося режима, он оказался жизнеспособным, посрамив политических аналитиков, предрекавших ему быстрый крах. Нет ничего нового под Луной, нечто подобное было в прошлом веке на Кубе. Подобно Кубе, Меганезию в шутку называют «Островами свободы». Эти страны похожи по численности населения и размеру сухопутной территории. Но Меганезия разбросана по тысячам мелких островов и атоллов Тихого океана, так что площадь ее акватория больше, чем площадь всей Африки. Такой вот парадокс. Режим здесь сильно отличается от кубинского, хотя он не менее, а скорее более репрессивный. Один остроумный комментатор назвал этот режим диктатурой без диктатора и анархией без анархистов. Еще один парадокс: по индексу благополучия Меганезия удерживается на 34 месте, немногим отставая от развитых стран. Туристу может показаться, что здесь полнейшая свобода, не ограниченная даже элементарными приличиями. Но стоит вам нарушить малейшее из правил местной великой хартии, как репрессивный аппарат обрушится на вас всей своей мощью. Это случилось недавно с несколькими гуманитарными организациями. Полиция без предупреждения открыла огонь по мирным манифестантам, десятки людей были ранены, двое убиты. Девятнадцать авторитетных общественно-религиозных лидеров были брошены за решетку и предстали перед судом. Их организации были запрещены, имущество – конфисковано, а сами они были приговорены к смертной казни, замененной затем на немедленную депортацию.

В чем же состояло преступление этих людей? Оказывается, они всего лишь потребовали уважения к религии и морали в том объеме, который гарантируется международными актами о правах человека. Подробнее о том, как понимают свободу в Меганезии, вам расскажет мой коллега, Майкл о’Доннел, находящийся сейчас в Страсбурге, где недавно завершилось скандальное выступление представителя верховного суда Меганезии. С вами был Кэн Уилсон, специально для CNN из Лантона.

1. Грендаль Влков, верховный судья по жребию.


…Чтобы не устраивать суету вокруг своей персоны, Грендаль в самолете не снимал широких тонированных очков, а в аэропорту пристроился в хвосте, за всеми пассажирами, и подошел к стойке пограничного контроля последним. Никто не обратил внимания на подтянутого дядьку, чуть выше среднего роста, лет 40, одетого в свободные штаны и рубашку из серого льняного полотна. У стойки очки пришлось снять для face-control.
Молодой лейтенант тут же утратил официальную флегматичность.
- Вы – тот самый Грендаль Влков? Судья?
- Да, а что, не похож?
- Я вас сразу узнал. Видел вчера по телевизору. Здорово вы их уделали!
- Правда?
- По-моему, да… Ваша карточка, сен Влков. С возвращением домой. Удачи!
- Спасибо, сен офицер.

До дома на атолле Сонфао было 700 километров, но домашняя атмосфера окружила его, едва он сел в трамвайчик, идущий из аэропорта к пирсам лантонской бухты. Пестрая публика всех рас и цветов кожи, одетая во все возможные фасоны одежды – от легких тропических джинсовых костюмов до традиционных саронгов и лава-лава, оживленно жестикулировала, болтая между собой и по мобильникам на всех восьми основных меганезийских языках. Заход солнца в Лантоне – бойкое время, вечерний час пик.

До объявления конкурса социальных администраций оставалось меньше месяца, так что рекламой политпрограмм – languen guangao, join our wishes, или «soc4u» (на языке SMS) были заполнены почти все светящиеся постеры – tusinbao вдоль трассы.

= Ты достаточно богат, чтобы не считать коммунальные расходы? А мы – нет. Голосуй за квоты для бедных в наблюдательном совете коммунальных служб. Это выгодно и тебе.

= Большинство профессий требуют двух высших образований. Вы хотите эффективной экономики? Голосуйте за социальную оплату второго высшего образования.

= Не хочешь жить в стране, где все похожи на тебя? Тогда голосуй за поддержку поселков островитян-аборигенов сегодня – ведь завтра ты не восстановишь их ни за какие деньги.

= Я мечтаю увидеть мир, но у меня не хватает денег. Вы же не хотите, чтобы через 10 лет вас окружали серые люди? Голосуйте за социальное финансирование детского туризма.

= Надоела дороговизна? Не устраивает качество товаров? Почему развитием Hi-Tech управляют неучи? Голосуйте за квалификационный ценз для социальных наблюдателей.

= Мы хотим видеть новое поколение здоровым? Тогда почему семьи с больными детьми получают большее социальное финансирование? Голосуйте за программу евгеники!

= Я работаю так же, как и ты и плачу те же социальные взносы. Тебе не стыдно делить мои деньги без меня? Голосуй за равное избирательное право для работающих подростков!

= 90 процентов преступлений могут быть пресечены гражданами. Голосуйте за включение полицейского минимума подготовки в программу школ – и безопасность станет дешевле.

= Тебе жалко денег вылетающих в трубу ускорителя или в космос? А тебе не жалко внуков, которым не хватит энергии и пространства? Наука – это будущее, голосуй за нее сегодня!

= Мы все любим живую природу – но не настолько же, чтобы возвращаться к пещерной жизни! Голосуйте за разумное ограничение экологических требований и расходов.

= Половина медицинских офицеров – шарлатаны! Половина медицинских препаратов вреднее, чем пирсинг и импланты! Долой медицинский контроль над салонами body-art

Рядом с этим постером болталось десятка два шумных не совсем одетых молодых людей. Их тела были раскрашены и декорированы блестящими аппликациями. Тут же стояла полицейская машина. Два полисмена, по виду – индус и ирландец, что-то выясняли у синеволосой молодой мулатки с серебряным кольцом в носу, одетой в набедренную повязку из люминесцирующей ткани.

Грендаль недовольно фыркнул. Он мог понять смысл раскрашивания тела, но пирсинг и прочее в таком роде совершенно не одобрял. Впрочем, каждый имеет право украшать свое тело так, как считает нужным - о вкусах не спорят…

У пирсов, лучами расходящихся от площади Че Гевары, покачивались сотни гидропланов разных конструкций, размеров и цветов, разрисованных эмблемами транспортных фирм, китайскими иероглифами, тотемными символами племен аборигенов-утафоа, и просто чем попало – в соответствии с художественными вкусами и фантазией владельцев. Еще несколько десятков таких же машин сновали в воздухе и катились по воде – либо взлетая, либо приводняясь. Акватория бухты была освещена множеством прожекторов и ярких реклам маршрутных авиатакси. Агенты профсоюза индивидуальных авиарикш, в основном подростки, фланировали по площади с транспарантами, изображающими направления полетов и расценки.

Иностранцы здесь, как правило, не могли сориентироваться и покупали билеты в офисе центрального агентства внутренних перевозок, стеклянная пирамида которого торчала посреди площади. Но Грендаль, как человек местный, за пять минут нашел дешевого авиарикшу до Сонфао. Дешевизна объяснялась, во-первых, наличием двух попутчиков летевших до атолла Тераруа (что для Грендаля означало крюк в полтораста километров), а во-вторых, отсутствием у авиетки публичного сертификата. О последнем обстоятельстве рикша-малаец, как положено, сообщил пассажирам – Грендалю и молодой парочке: китаянке в почти невидимом бикини и русскому в ярко-оранжевых шортах-багамах.
- Da huya sya (ну, здорово) – иронично сказала девушка и полезла на заднее сидение
- Po huy (без разницы) – лаконично добавил ее кавалер, и последовал за ней.
Грендаль пожал плечами и уселся рядом с пилотом. Любой меганезиец знает, что рикши плюют на сертификацию. Машина, соответствующая стандартам, на порядок дороже, чем простой стеклопластиковый «fly-wing» с компактной, но мощной спиртовой турбиной.

Рикша убедился, что пассажиры пристегнулись, захлопнул обтекатель и пробурчал что-то в микрофон. Потом включил взлетные огни и сразу следом – турбину. Авиетка пробежала сотню метров по воде и круто взмыла в воздух. Некоторое время Грендаль смотрел вниз, на океан, усеянный россыпями ярких разноцветных огней – кругом проходили морские трассы и зоны лова рыбы или планктона. В какой-то момент он потерял границу между океаном и звездным небом, и нечувствительно задремал на два часа – до самой посадки в лагуне Тераруа. Рикша подрулил к одному из пирсов, парочка сошла, а на заднее сидение залез дедушка лет 80, по виду – чистокровный утафоа.
- До Рагаиу, - проворчал он.
- Через Сонфао, - ответил рикша, - двадцать фунтов.
- Пятнадцать, - сказал дедушка
- Семнадцать, - скинул рикша.
Дедушка кивнул, не торопясь, отсчитал купюры, передал их рикше и начал флегматично набивать табаком-самосадом длинную украшенную затейливой резьбой трубку.
Авиетка развернулась к горловине лагуны и снова взмыла в воздух.
Через десять минут по салону поплыли клубы ароматного дыма. Пилот пару раз чихнул и приоткрыл посильнее вентиляционные жалюзи. Вообще-то курить тут не полагалось, но в Меганезии было не принято делать замечания старикам по такому мелкому поводу.

Грендаль вытащил мобильник и ткнул иконку с изображением вигвама.
- Хай, дорогая!
- Алоха! Ты где?
- Милях в ста к зюйду. Через полчаса приводнюсь.
- OK. Иржи встретит тебя на боте. Есть хочешь?
- Чертовски!
- Это хорошо. Люблю тебя кормить.
- А я вообще тебя люблю.
- Я тебя тоже. Жду – целую.

- Жена? – осведомился дедушка с заднего сидения.
- Да.
- Красивая?
- Очень, - ответил Грендаль. Он придерживался твердого убеждения, что Лайша – самая красивая женщина, по крайней мере, в пределах нашей галактики.
- Детей много?
Грендаль молча показал два пальца.
- Уах! – возмутился дедушка, - никуда не годится. Сильный мужчина, красивая женщина, надо делать много детей. Кто будет жить под Луной, если вы такие ленивые?
- Мы работаем над этим, - дипломатично ответил Грендаль.
Старый утафоа пробурчал что-то и отвернулся к окну. Видимо, такой ответ не развеял его опасений по поводу численности будущего поколения.

Через некоторое время вдалеке появились мерцающее пятно света: в маленьком Сонфао-сити бурлила ночная жизнь. Вскоре стали видны огоньки домов вдоль берега и желтые точки мачтовых фонарей на рыбацких проа, промышляющих вокруг атолла. А потом запищал мобильник.

- Привет, па! Два румба к зюйду это твой прожектор?
- Привет, Иржи. Думаю, да, вроде рядом никто не летит.
- Ага! Я уже в лагуне, сейчас зажгу красный фейер.

Секунд 15 - и почти посреди лагуны вспыхнула ярчайшая алая звездочка. Грендаль тронул рикшу за плечо и показал туда пальцем.
- Встречают? – спросил тот, слегка сдвинув наушники.
- Да. Сын.
- О! Сколько ему?
- Тринадцать.
- Вы разрешаете парнишке ночью одному водить бот в океане?
- И хорошо! – встрял дедушка, - я с десяти лет ходил по ночам между атоллами.
- Это хорошо для утафоа, - возразил пилот, - у вас моряцкий навык в генетике.
Дедушка ехидно хмыкнул
- Сказал научное слово и думаешь, все объяснил?
- Лагуна, все же, не открытый океан, - заметил Грендаль.

Авиетка чиркнула поплавками по воде, описала длинную дугу и закачалась на слабой волне в сотне метров от маленького бота. Пилот откинул обтекатель.
- Счастливо!
- Удачи в небе! - ответил Грендаль, вылез из кабины на правый поплавок, а оттуда перепрыгнул в бот, уже успевший подойти вплотную к ним.

Иржи с серьезным видом сидел за штурвалом. Худощавый, смуглый, он был бы похож на аборигена, если бы не рыжие волосы, зеленые глаза и характерные веснушки, которые не мог до конца скрыть даже загар.
- Они задолбали, да, па? – спросил он, разворачивая бот к дальнему пирсу.
- Кто? – спросил Грендаль.
- Ну эти, - мальчик покрутил левой рукой в воздухе, - западные оффи. Наш препод по экостории говорит: они – козлы и всегда были козлы. Как юро и янки при них живут?
- Он так и говорит?
- Ну, не совсем, но по ходу так. А что, неправда?
- Как тебе сказать, - Грендаль почесал в затылке, - конечно, политики там гниловатые. Но люди как-то привыкли. Живут себе, а этих воспринимают как привычную неприятность. А как у нас тут?
- Нормально. Мы с Саби вчера на ветряке турбину поменяли. Пока ма была на работе.
- Ты что, таскал Саби на мачту? Ты вообще понимаешь, что она еще маленькая?
- А что? Она сама захотела, а я виноват да?
- Страховочные пояса хоть надевали?
- Спрашиваешь… Только ма все равно ругалась.

Они уже приближались к дому. Сам дом, как обычно в меганезийском субурбе, состоял в основном из террас, балконов и навесов. Только в глубине была железобетонная коробка, обвитая деревянной лестницей и накрытая пластиковой крышей в форме расправившей крылья бабочки. У бабочки был хоботок, точнее - шланг, опущенный в бассейн: крыша служила конденсационным водосборником и солнечной батареей. По бокам торчали: мачта ветряка – генератора, штанга со спутниковой тарелкой – антенной и кронштейны с баками локального водопровода. Такая автономность жилья была здесь обычным делом. Многие даже топливный спирт гнали на заднем дворе, из перебродивших водорослей. Влковы предпочитали покупать не только горючее, но и рыбу, на рынке в сити, отчего слыли на Сонафо людьми не особо хозяйственными. Ладно – спирт, но что за каприз – покупать рыбу, когда ее полно в океане? А вот фруктовый сад Влковых был предметом некоторой зависти соседей. Почему, спрашивается, у них растут не только обычные местные штуки вроде тыкв и бананов, но даже виноград, из которого получается отличнейшая водка-граппа? Никто не верил, что это только следствие агроинженерной профессии Лайши, и связывали ее талант с италийским происхождением. Все, мол, дело в генах… От фасадной террасы к океану спускалась широкая лестница, проходящая через еще один навес на пирс. Под навесом стоял обычный набор: дешевая авиетка и маленький внедорожник. У пирса был пришвартован проа – не солидный, рыболовный с лебедкой для трала, как у большинства, а легкий, спортивный. Баловство, одним словом.

На оконечности пирса, между двумя габаритными маячками, уперев руки в бока, стояла Лайша собственной персоной. На ней были шорты и майка, имевшие когда-то белый цвет, а сейчас – пятнисто-сиреневые от фруктовых пятен. До образца калабрийской фермерши она не дотягивала по объему груди и бедер, да и высшее образование было тут некстати. Но, если уж Лайша решала войти в эту роль, то такие мелочи никак не могли ей помешать.

- Ужас! – заявила она, окинув мужа насмешливым взглядом ярко-зеленых глаз - Щеки впалые, лицо зеленое. Что, черт возьми, ты ел в этой варварской Европе? Срочно за стол!
- Уф, - Грендаль обнял ее, зарывшись лицом в жесткие волосы цвета темной бронзы, - За стол это здорово. Если мне еще нальют стаканчик граппы…
- Нальют, когда ты примешь душ и бросишь тряпки в стиральную машину. Похоже, ты собрал всю пыль с этого грязного континента.
- Ничего подобного, - возразил он, - там осталось предостаточно.
- Тогда я рада. Европейцам не придется менять свои привычки. А сейчас марш в душ.



2. Великая Хартия и мировая пресса.


Вымывшись и завернувшись в лава-лава, Грендаль наконец-то почувствовал себя действительно дома. Вся семья собралась на центральной террасе, выполнявшей по обычаю функции гостиной. Правда, Иржи уже сидел за компьютером и что-то делал в интернете, а Саби спала, завернувшись в плед, в широком кресле перед выключенным телевизором в дальнем углу террасы, выходящем в сад.
- Опять смотрела мультик про этих дурацких белых медведей? – спросил он, почесав ее за ухом.
- Они не белые медведи, а панды, - пробурчала она, не открывая глаз.
- Ты уверена, что это меняет дело?
- Меняет. Они не дурацкие, а прикольные, - она все-таки открыла глаза, - Ой, па, а ты когда приехал?
- Минут десять назад. Милая, тебе не кажется, что в детской тебе было бы удобнее спать, чем здесь? Мы ведь будем шуметь и все такое…
- Шумите, - великодушно разрешила она, поворачиваясь на другой бок, - мне не мешает. А в детской мне скучно.

- Да что ты, в самом деле, - вмешалась Лайша, - пусть девочка спит, где ей удобнее, какая разница. И вообще садись за стол. Я тебе налила айн-топф, его надо есть горячим.
- Айн – что?
- Айн-топф. Суп такой баварский.
- А-а, - сказал он, подходя к столу, - пахнет вкусно.
- Вот и ешь уже, - сказала Лайша, - и, кстати, выкладывай, что там было? По ТВ это походило на цирк шапито. Я ничего не поняла и выключила.
- Я тоже не понял, - ответил он, проглотив первую ложку супа, - надо было ехать Джелле или Макрину. Или, хотя бы, Ашуру. В конце концов, это они судьи по рейтингу, а я - по жребию. Вот и объясняли бы…
- А Макрин вчера звонил и говорил: правильно, мол, что тебя отправили.
- Что еще говорил?
- Говорил, они твои дела на завтра расписали между собой, так что у тебя выходной. Как бы, подарок от коллег.
- Очень мило с их стороны, - буркнул Грендаль с набитым ртом.
- Не ворчи, Грен.
- Я и не думал ворчать, - возразил он, - кстати, где граппа?
- Слева от тебя в пластиковой бутылке.
- А, вижу, - он наполнил рюмку и демонстративно облизнулся.
- Па, что такое «фашист»? – спросил Иржи.
- А в энциклопедии лень посмотреть?
- Ага, там написано, что фашисты – преступники, они создали государство, запретили оценивать администрацию, и убивали всех, кто хотел регулировать общество иначе, чем они. А еще они устроили войну, хотя на них никто не нападал.
- Ну, в общих чертах, правильно написано.
- Па, а почему тогда в «Europe monitor daily» написали, что ты – фашист?

Лайша повернулась к Грендалю и, разведя руками, сказала.
- Угораздило же тебя с этим жребием. Придется объяснять ребенку про фашистов.
Грендаль пожал плечами, отхлебнул чуть-чуть водки и спросил:
- Иржи, ты ведь знаешь, из-за чего я летал в Страсбург?
- Потому что ты выгнал из страны каких-то пидорасов, а другие пидорасы из-за этого подняли крик.
Лайша всплеснула руками:
- Эй, от кого ты услышал это слово?
- От тебя, ма, - невозмутимо ответил мальчик, - ты так объясняла дяде Ван Мину. А кто такие пидорасы?
- Это те, кто из сексуальной ориентации делает политику, - вмешался Грендаль, - но давай-ка сперва разберемся с фашистами. Во-первых, решение о депортации принял не я один, а коллегия верховных судей, выбранная на этот год. Ты ведь знаешь как…
- Знаю, - перебил Иржи, - это же в первом классе проходят.
- Вот и молодец. А теперь распечатай-ка газету, где написано, что я - фашист.

Иржи несколько раз щелкнул мышкой. Из принтера выползло несколько листов. На первом был яркий заголовок:

«Шокирующие заявления инквизитора Меганезии». Ниже была Фотография Грендаля и комментарий к ней: «Впервые в истории трибуна евросоюза предоставлена фашисту» сказал Нурали Абу Салих, комиссар совета Европы по правам человека».

Дальше шел текст, в котором фрагменты прямой речи Грендаля были выделены жирным курсивом. Подборка была впечатляющая - журналисты хорошо поработали ножницами.

«Великая Хартия выше всех моральных авторитетов и всех религий со всей их историей»

«Мы вправе подвергнуть моральному террору любую группу лиц с особыми обычаями»

«Если каким-то людям не по нраву наши порядки - пусть убираются из страны»

«Никаких компромиссов. Суд выносит постановление, а полиция должна его исполнить»

«Ваша демократия – декоративный платочек, скрывающий ошейник раба на вашей шее»

«Вас поставит на колени любой знающий, что ваша толерантность – это просто трусость»

«Правительство не вправе позволять себе такие излишества, как совесть и милосердие»

«Мы содержим эффективный военный корпус, чтобы он применял оружие без колебаний»

- Ничего себе, - заметила Лайша, заглядывая ему через плечо, - хорошо еще, тебя только фашистом обозвали, а не каннибалом, например. На каннибала это вполне тянет. Грен, ты что, правда, вот так и говорил?
- Да нет же, Лайша… В смысле, я действительно это говорил, но не просто так. Все это было в тему. А здесь из меня нарезку сделали. Типа китайской лапши. Вот, черти…
- По телеку все было классно, - поддержал его Иржи, - ты зря не смотрела, ма. Я, между прочим, записал. Ну, для истории, и вообще, вдруг пригодится.
- Но это было невозможно смотреть, там же все подряд, а слушать бред, который несут эти надутые идиоты…
- Они прикольные, - возразил мальчик, - говорят по-английски, а про что - непонятно.
- Это называется «политическая риторика», - объяснил Грендаль, - такой специальный прием, чтобы заморочить голову слушателям и отвлечь их на всякую чепуху.
- Ты устраиваешь то же самое, когда тебе лень делать уроки, - добавила Лайша.

Тренькнул телефон.
- Начинается, - вздохнул Грендаль, - алло, слушаю… А, привет мама… Нормально… Да, нет, не особо… Ну их всех… Нет, честно, не хочу и не буду… Ну, дай ему трубку… Да, папа… Подожди, минуту, я тебе объясню…

Объяснял он четверть часа. Потом положил трубку и молча налил еще рюмку граппы.

- Что там? – спросила Лайша.
- Предки тоже прочли про «фашиста», - сообщил он, - папа уговаривал меня написать по e-mail координатору иностранных дел, чтобы в европейский комиссариат отправили ноту протеста, а Абу Салиху и этим журналистам закрыли визы в Конфедерацию.
- А они по контракту обязаны это делать?
- Угу, - ответил он, хлебая остывший айн-топф, - папа даже нашел там параграф номер такой-то, обязанности в случае враждебных действий иностранных должностных лиц по отношению к общественным офицерам Конфедерации. Только зачем? Ну, закроют этим уродам визы, а дальше? Это как сквалыжничать по поводу чужой кошки, нагадившей на твое крыльцо. Отсудишь полста монет, зато все будут знать, что тебя обгадили.

Снова тренькнул телефон.
- Доедай, Грен, я поговорю, - бросила Лайша, - алло, кто это?… Чего?… А, понятно… Я – Лайша, люди говорят, что я - его жена, может, это и правда… Чего?… Вообще-то он суп ест… А это обязательно сегодня?… Понятно. А по телефону никак?… Ах, согласовать текст? Ладно, сейчас спрошу.
- Кто там на нашу голову? – поинтересовался Грендаль.
- Пресса. Парень из «Pacific social news» напрашивается в гости. Говорит, что в твоем контракте сказано…
- Знаю. Там сказано «незамедлительно». Баран, не мог к самолету подъехать. Он бы еще ночью напросился.
- Так что, пусть приезжает?
- Ну, да, а что делать.
- Приезжайте, - сказала Лайша в трубку, - встретим вас в лагуне, только позвоните за 20 минут… Ах сами?.. Гм, ну, если последняя модель… Ладно, если заблудитесь – звоните.
- Что он такое сказал? – поинтересовался Грендаль.
- Что у него какой-то навороченный спутниковый прибор наведения, в общем, посмотрим. Его зовут Малик Секар. Он обещал быть через час с минутами.
- Это журналист? – спросил Иржи.
- Да, сынок.
- Надеюсь, он будет не такой прожорливый, как тот, что был в прошлый раз. Тот слопал весь абрикосовый джем. Как у него жопа не слиплась…



3. Малик Секар, репортер «Pacific social news».


… Реактивный гидроплан военного образца садился на такой скорости, что у зрителей возникли серьезные опасения сначала – за судьбу репортера, а затем, за судьбу своего пирса, который тяжелая машина чуть не снесла при торможении. Тем не менее, ничего особенного не произошло, поплавки только слегка шаркнули по настилу, и из кабины почти тут же выпрыгнул молодой спортивного вида парень, вероятно индонезиец, лет 30, если не меньше. Он сразу улучшил мнение Иржи о репортерах, поскольку притащил большущий фруктовый торт.
- Извините, что на ночь глядя, - сказал он, пожимая руку Грендалю, - я подумал, вдруг у вас нет ничего к чаю и...
- Хорошо же вы обо мне думаете, - перебила Лайша.
- Простите…
- Ерунда, я пошутила, - снова перебила она, - берите свое служебное барахло и садитесь за стол. Водки хотите, сен Секар?

Секар глянул на одиноко стоящую в углу стола полупустую рюмку и кивнул.
- Немножко. Если очаровательная сен Лайша составит…
- Составлю, - фыркнула она, доставая еще две рюмки, - сколько стоит та зверская штука, на которой вы прилетели?
- Не знаю, это служебная, - ответил он, водружая на стол торт, ноутбук и небольшую видеокамеру, - редакция купила ее у морской патрульной службы, когда те обновляли матчасть. Для патруля она устаревшая, а для прессы нормально.

Иржи, тем временем, по-хозяйски подвинул торт поближе к себе.
- Смотри, не обожрись, - предупредил Грендаль, и, повернувшись к репортеру, сделал серьезное лицо, - я готов, сен Секар. Поехали?.. А что вы такое уже стучите?
- Introduction, first impression - сообщил тот, с невероятной скоростью шлепая пальцами по клавиатуре, - так обычно делают, если беседа проходит в домашней обстановке… А кто вы по профессии, сен Влков?
- Я закончил колледж по автоматизированной бытовой технике, а вторая специальность – техническая экспресс-диагностика. По работе решаю проблемы потребителей со всякими генераторами, компьютерами, холодильниками, микроволновками, и прочим в этом роде.
- Ваша профессия помогает в деятельности судьи?
- Как сказать. С одной стороны – да, опыт работы с рассерженными людьми и все такое. С другой стороны, из-за этого коллегия и выбрала меня, чтобы ехать в этот долбанный… В смысле…
- Я понял куда, - сказал Секар, - продолжайте, это очень интересно.
- Ничего интересного. Всех трех экспертов по социальному регулированию отклонили. Ашура и Макрина потому что они, видите ли, заумные, а Джеллу - за некоторую резкость суждений. Тин Фан отклонили за то, что она программист и не имеет опыта работы с людьми, а Дольфина – за то, что он имеет слишком специфический опыт. Он судовой механик, у него такой сленг…
- Вы тоже за словом в карман не лезли, - заметил репортер.
- Да, но я держался в рамках нормативной лексики, если вы понимаете, о чем я.
- Вполне понимаю. А как вы оцениваете действия социальной администрации в этом инциденте? Я имею в виду, историю с этим фильмом, с которого все началось?

Грендаль вздохнул и наполнил рюмки. Сделал маленький глоток. Почесал затылок.
- Если вкратце, сен Секар, я считаю, что полиция допустила массовые беспорядки, вместо того, чтобы пресечь их превентивно.
- Но деятельность полиции связана правилом Великой Хартии о невмешательстве в частную жизнь, - заметил Секар.
- И что? – возразил Грендаль, - Раз на этих условиях их фирма участвовала в конкурсе на эту область администрирования, значит, должна была предусмотреть такие сложности. За это общество им и платит, верно?
- Они предупреждали правительство о возможном социальном недовольстве прокатом фильма «Дети троглодитов», - напомнил репортер, - они предлагали ряд превентивных мер общего характера и...
- Я высказал свое мнение, - перебил Грендаль, - и я думаю, что на полицию будет наложен штраф. Они подписывались не на что-то там общее, а на обеспечение конкретной безопасности. Но официально это дело ведет Джелла Аргенти, и лучше спросить у нее.
- Да, я знаю, я уже договорился о встрече с сен Аргенти.

Грендаль улыбнулся:
- Готов спорить, что она назначила Вам свидание в час ночи, в клубе рок-спорта, на острове Акорера.
- Да, а как вы угадали?
- Просто за 4 месяца можно узнать некоторые устойчивые привычки коллег.
- Тогда, понятно. А что вы скажете о самом фильме? Вы смотрели?
- Смотрел. 8 реальных историй о сексуальном опыте наших школьников. В стиле Ромео и Джульетты. Лейтмотив: семьи фундаменталистов - это источник трагедий. Женщина выплескивает в лицо девушке-подростку серную кислоту, потому что она «блудница» и «совратила» ее сына. Мужчина стреляет из ружья в восьмиклассника, который «растлил невинность» его дочери. Другой мужчина бросает самодельную бомбу в подростков на пикнике «nude-stile», потому что они «склоняют одноклассников к греху». И так далее.
- Вы согласны, сен Влков, что фильм возбуждает ненависть к патриархальным семьям?
- Скорее к их укладу. Впрочем, это не важно. Режиссер вправе показывать проблемы общества так, как он их видит. Если бы он призвал к физической расправе с этими семьями, то нарушил бы Великую Хартию, но он только дал моральную оценку.

Репортер задумчиво покрутил в руке рюмку и залпом выпил. Очень своевременно, поскольку Лайша уже поставила на стол большой китайский чайник и четыре чашечки из полупрозрачного фарфора.
- Спасибо, сен Лайша, вы очень…
- Продолжайте, мальчики, - перебила она, - это все безумно интересно.
- Я предвижу ваш следующий вопрос, сен Секар, - сказал Грендаль, - как быть, если эта моральная оценка превратилась в обоснование морального террора против определенного стиля жизни, семейного уклада, религии, убеждений? Я угадал?
- В общем, да, - признал репортер, - я имею в виду аргументы представителя Ватикана.
- Тогда я отвечу вам так же, как ответил ему. Великая Хартия запрещает контроль актов морального выбора. Мы вправе подвергнуть моральному террору любую группу лиц с особыми обычаями, неприемлемыми для свободных людей. Эта группа вправе ответить нам тем же. Правительство не может сюда лезть, а обязано только пресекать насилие и угрозы его применения. Таково правило о невмешательстве в частную жизнь, верно?

Репортер улыбнулся и кивнул.
- Конечно. Но, как мы помним, Абу Салих привел контраргумент: Великая Хартия - это учение этического нигилизма. О какой свободе морального выбора можно говорить, если одно из этических учений объявлено высшим законом и обеспечено правительственным принуждением?
- Этому типу я отвечал длинно, вам отвечу коротко и наглядно. Человек имеет право свободно владеть своим имуществом, верно?
- Согласен. Но какое…
- Эта камера – ваше имущество? – перебил Грендаль.
- Да, и что?

Грендаль подмигнул ему, взял камеру со стола и положил к себе на колени.
- Вот так. Теперь она моя, и я свободно ей владею. Есть возражения?
- С чего это она ваша, сен Влков?
- С того, что она у меня, вы же видите.
- Но она у вас потому, что вы ее у меня отняли, - возразил Секар.
- Вы зовете полицию, - констатировал Грендаль, - Иржи, будь другом, сыграй полисмена.


4. Иржи Влков, меганезийский школьник.


Мальчик вытер измазанную кремом физиономию, наставил на Грендаля указательный палец и строгим голосом заявил:
- Вы арестованы за грабеж! Верните эту вещь владельцу и следуйте за мной!
Грендаль быстро вернул камеру на стол, поднял руки вверх и пояснил.
- Вот видите, сен Секар, в чем различие между владением своей вещью и владением чужой? Так и с моральным выбором. Он принадлежит личности, и личность может распорядиться им так и этак, как захочет и когда захочет.
- В том числе, сделать выбор в пользу патриархальной морали, - вставил Секар.

Грендаль энергично кивнул.
- Да. Но только за себя, а не за соседа. Если личность принуждает соседа к своей морали, то присваивает чужое право. Как я, в случае с вашей камерой. Никто не говорит, что запрет отбирать чужую вещь - это нигилистическое отношение к владению, верно? Говорят наоборот: что это – защита права владения. Такую же защиту Великая Хартия обеспечивает праву на моральный выбор. При чем тут нигилизм?
- Гм, - задумчиво сказал репортер, - все это очень наглядно, но есть существенная разница. В отличие от свободы владения, свобода морали ограничивается социальными нормами. Я имею в виду, запрет на общественно-опасные поступки, на тот же грабеж, в частности.
- Никаких отличий, - спокойно ответил Грендаль, - то же самое касается владения вещами, которые, находясь в частных руках, создавали бы угрозу для всех. Люди договариваются, чтобы частные лица не владели атомными бомбами или национальными электросетями.
- Есть страны, где национальные электросети находятся в частных руках, - заметил Секар.
- В этих странах и грабят безнаказанно, - парировал Грендаль, - причем именно те, в чьих руках электросети. Попробуйте их наказать. Они вам электричество выключат – и все.
- Ага, - сказал репортер, - я попробую сформулировать. Значит, возможность навязать окружающим свою мораль, так же опасна, как частное владение атомной бомбой?

- Грен, ты этого не говорил, - вмешалась бдительная Лайша, подливая всем чая.
- Я помню, милая. Хотя, то, что сказал сен Секар, кажется мне правильным.
- Просто я хотел перейти к вопросу о мере наказания, - пояснил репортер, - ведь, если смотреть практически, то патриархально настроенные граждане всего лишь нахулиганили в нескольких магазинах, клубах и кинотеатрах. Обычно за это бывает штраф и небольшой срок лишения свободы, ведь так?
- Именно так, - подтвердил Грендаль, - но их преступление состояло не в хулиганстве, а в попытке запугать граждан и навязать правительству представления своей социальной группы. А как это называется, знает даже ребенок.

Иржи оторвался от чая и выпалил.
- Это называется «тирания» и карается высшей мерой гуманитарной самозащиты
И пояснил свои слова недвусмысленным жестом, завязав узел воображаемой веревки.
- Ого! – изумился Секар, - откуда такие познания?
- Будто вы в школе не учились, - в свою очередь, изумился мальчик.

Репортер задумчиво поскреб щетинистый подбородок
- Не знал, что теперь этому учат в школе…
- И правильно делают, что учат, - вмешалась Лайша, - мы в свое время нахлебались всяких там высших интересов нации, и нечего нашим детям наступать на эти грабли.
- К счастью, Великая Хартия позволяет заменить смертную казнь депортацией, - разрядил обстановку Грендаль, - мне бы очень не хотелось приговаривать девятнадцать человек к лишению жизни.
- А если бы не существовало такой альтернативы, как депортация? – спросил репортер.
- Эта альтернатива придумана еще древними эллинами. Мало ли по каким путям могла бы пойти история? Это беллетристика, а у нас - реальность.

Секар улыбнулся и развел руками.
- Хорошо, сен Влков, давайте вернемся к реальности. Как вы прокомментируете заявление представителя Human Rights Watch о том, что в Конфедерации создана – цитирую по памяти – «обстановка тотального глумления над идеалами религиозно-культурных общин, чья мораль и чьи взгляды отличаются от правительственных»?
- О реальности, так о реальности, - согласился Грендаль, - давайте мысленно перенесемся на какой-нибудь из ближайших крупных островов. Ну, например, Нукуалофа. И заглянем в первое попавшееся открытое кафе у берега океана. Что мы увидим?
- Ничего особенного, - предположил репортер, - люди кушают, пьют напитки, или там…
- Мы увидим, - перебил Грендаль, - совершенно разных людей, отдыхающих согласно своему вкусу, но при этом соблюдающих необходимый минимум общих правил. Кто-то может сидеть там во фраке, кто-то – в купальнике, кто-то - в лава-лава, а кто-то – вообще голым. Это – личное дело каждого. Но никто не вправе ломать мебель или нападать на других посетителей и каждый должен платить за то, что съел и выпил. Это так, верно?

Репортер кивнул и Грендаль продолжил.
- При этом, конечно, одним людям может не нравиться внешний вид или стиль поведения других. К примеру, пуританина будут смущать раздетые натуралисты, натуралистам не понравятся мусульмане, закутанные с ног до головы в темную ткань, а мусульманам будет неприятно, что у большинства женщин открыты лица, а у многих – и другие части тела.
Каждый может поспорить с другим о вкусах и приличиях, но другой вправе остаться при своем мнении и даже вообще отказаться обсуждать эту тему, если ему не интересно. Но никто не должен навязывать свой вкус другому. Если пуританин начнет силой натягивать на натуралиста костюм, а тот начнет сдирать с пуританина одежду, то будет черт те что.
- Все так, - согласился Секар, но что, если кого-то так оскорбляет внешний вид другого, как если бы тот ударил его по лицу? Не лучше ли пойти на некоторые компромиссы?
- Не лучше, - ответил Грендаль, - граждане не обязаны терпеть неудобства из-за чьих-то неврозов, а нервного гражданина никто не заставляет бывать в общественных местах
- Это если речь идет о пляже, - возразил репортер, - а как на счет места работы или учебы?
- Там надо работать или учиться, а не глазеть на коллег, - отрезал Грендаль, - и вообще, как сказал Ганди, пусть каждый занимается своими делами и предоставит другим заниматься своими. Иначе никакое социального регулирование не поможет… Иржи, если ты намерен и дальше играть в doom, то или иди на второй этаж, или убавь звук.

Мальчик обиженно фыркнул и повернул тюнер, так что грохот пулеметных очередей ослаб примерно до уровня треска цикады.
- Вообще-то тебе пора спать, - заметила Лайша.
- Но ма, я тоже хочу послушать.
- А ты не боишься проспать завтрак? Учти, трое одного не ждут.
- Я поставлю будильник.
- Ладно, но я тебя предупредила.

- Кстати, о детях, - сказал репортер, - родители вправе воспитывать детей в той системе ценностей, которую считают правильной. Это записано в Великой Хартии.
- Сейчас посмотрим, - Грендаль встал и снял с полки тоненькую книжку, - так, вот тут у нас про семью… Ага, читаю. «Частные лица, на чьем иждивении дети находятся в силу родства, вправе свободно выбирать этическую систему для их воспитания, но лишь такую, какая не обрекает детей на заведомые страдания и не противоречит общей безопасности».
Извините, но деятельность судьи требует педантичности. Вы, сен Секар, сказали неточно.
- Не могу сказать, сен Влков, что мне полностью понятно то, что вы сейчас прочли.
- На самом деле, этот пункт очень прост для понимания, - заметил Грендаль, - мне его объясняли на примере истории с аборигенами-островитянами. Всего четверть века назад большинство из них вынуждены были жить в резервациях. Не потому, что их кто-то не выпускал, а потому, что они понятия не имели о том, как жить в техногенной обстановке. В лучшем случае они сразу попадали в полицию за мелкие кражи - они ведь понятия не имели о частной собственности. Хуже то, что они ничего не знали о дорожном движении, электричестве, бытовой химии… Обычные предметы, среди которых мы спокойно живем с самого детства, становились для аборигенов-островитян убийцами…
- Но, сен Влков, - перебил репортер, - политику ассимиляции никак нельзя назвать безупречной. Почему было не позволить им жить в резервациях, как они привыкли?
- Вы сами-то поняли, что сказали? – вмешалась Лайша, - средняя продолжительность жизни в резервациях была тридцать лет, а каждый десятый ребенок до года не доживал! А ведь аборигены – такие же люди, как европейцы, индокитайцы или англосаксы.
- И такие же граждане Конфедерации, как все мы, - добавил Грендаль.

- Они становятся такими же, как мы - уточнил Секар, - но при этом их культура исчезает.
- Что?! – возмущенно воскликнул Иржи, - Ато утафоа иинэ ла каа то ируо аноотари!
- Э… - смущенно протянул репортер, - а что это было?
Мальчик снисходительно фыркнул и перевел:
- Народ утафоа не исчезнет, пока светят луна и солнце. Культура не чья-то, а наша общая! Как небо или океан.
- Молодец! – сказала Лайша, потрепав сына по голове.
- Эитона-тона раа ле, - согласился Грендаль.
Секар чуть не уронил чашку.
- Что вы сказали, сен Влков?
- Я сказал: «вот слова настоящего человека». Это серьезная похвала.
- А откуда вы знаете язык аборигенов?
- Это второй официальный язык Конфедерации.
- Я знаю. Но мне казалось, это просто формальность…
- Ничего подобного. Он восемь лет как введен в школьную программу. Лайша и я выучили его вместе с Иржи, только и всего. Кстати, очень красивый язык.



5. Ваша толерантность – это просто трусость


Репортер демонстративно поднял руки вверх.
- Сдаюсь, сен Влков! Проблема культуры аборигенов снимается.
- Пока еще не снимается. Есть проблема сохранения особых ремесел и изящных искусств, связанных с бытом. Не так просто включить самобытные поселки утафоа в современный субурб… Но мы несколько уклонились от темы, да?
- Да, действительно… Мы говорили о патриархальных семьях в другом смысле. Я имею в виду, что у их детей нет той проблемы, которая была у детей аборигенов.
- Как же нет? – возразил Грендаль, - проблема та же самая. Дети из патриархальных семей не умеют жить в той информационной обстановке, которая есть в техногенном обществе. Вы сами говорили: для выходца из патриархального круга чей-то внешний вид - это как удар по лицу. Ребенок с патриархальным воспитанием приходит в школу – и с порога получает как бы серию пощечин. Теперь вернемся к тому пункту Великой Хартии…

- Подождите, не так быстро! – взмолился Секар, - что бы ни было написано в этом пункте, основа Великой Хартии в том, что никто не может совершать произвольное насилие над человеком!
- Произвольное объективное насилие, - уточнил Грендаль.
- А ударить по лицу – это не объективное насилие?
- Ударить по лицу – это объективное насилие. А действие, которое только для данного конкретного человека все равно, что удар по лицу – нет. Объяснить подробнее?

Секар кивнул головой, не отрываясь от ноутбука. Его пальцы летали по клавиатуре.
- Я объясню так, как объясняли мне, - сказал Грендаль, - возьмем индивида, который испытывает страдания, если кто-то наступил на его тень. В некоторых племенах тень считается частью организма, так что пример жизненный. Что нам теперь, исходить из этого обычая и защищать человеческую тень так же, как тело?

- Это неудачный пример, - сказал репортер, - какое-то вздорное суеверие…
- Именно поэтому пример удачный. Действия объективно не затрагивают тело человека, но он приравнивает их к физическому насилию. Чтобы учесть такие суеверия, придется урезать свободу передвижения людей, совершить над ними объективное насилие.
- Ладно, пусть будет ваш пример. Конечно, специально защищать тень – это вздор, но, с другой стороны, специально наступать на тень человека, который придает этому значение, как-то нехорошо. А как отмечал доктор Ахмади в своем выступлении...

Грендаль устало потянулся и зевнул
- Ну, конечно. Этого поросенка раздули до размеров слона.
- Почти что, - согласился репортер, - где-то метра три в диаметре.
- Нет, сен Секар, я имею в виду первого поросенка, с которого все началось.
- Боюсь, я не совсем в курсе, сен Влков…
- Сейчас расскажу. Все началось в школе. Одна семья попросила учителя запретить в классе, где учился их ребенок, авторучки с изображением поросенка из популярного мультфильма. Они были мусульмане, а у них особые табу в отношении свиньи. Учитель сказал, что такие вещи находятся в компетенции родителей. Тогда отец ребенка поднял вопрос о поросенке на родительском собрании, но сделал это недостаточно тактично. В результате ему пригрозили полицией, а об инциденте стало известно всем школьникам. Через несколько дней остальные дети пришли в классе в футболках с большим рисунком того же поросенка, да еще наклеили стикеры с тем же поросенком на все, что можно. У ребенка-мусульманина случилась истерика, а мусульманская община обратились в суд с заявлением об истязании и дискриминации. Суд опросил учителей и школьников, но не обнаружил объективных действий, которые могли бы так квалифицироваться. Разумеется, суд доставил неудобства детям и их родителям, что и вызвало, по выражению прессы «свиной бум». Пиком, как вы знаете, стали огромные резиновые свиньи, надутые гелием - многие жители подняли их над своими домами, кафе и лавками накануне Хэллоуина.
- Из-за чего и произошли стычки, потребовавшие вмешательства полиции, - добавил Секар, - разумно ли было доводить до этого?
- Разумно ли с чьей стороны? – спросил Грендаль.
- Я имею в виду, может, лучше было пощадить чувства этого мальчика и уступить в такой мелочи, как детские авторучки? Свет что ли клином сошелся на этом поросенке?

Возникла пауза. Грендаль на четверть минуты задумался, а затем сказал:
- Авторучки - детские, а проблема - взрослая. Свет всегда сходится клином на какой-то мелочи: картинках, футболках, воздушных шариках. Из этих мелочей складывается наша свобода. Мы учим детей быть свободными именно на таких мелочах. Я прочел в одной старой книжке: свобода – это возможность открыто делать то, что кому-то не нравится. По-моему, очень правильная мысль.
- А вы не боитесь, что таким путем мы отучим детей от милосердия?
- Не боюсь. К милосердию не принуждают - так я ответил доктору Ахмади. Милосердие это стремление опекать и защищать, а не подчиняться и терпеть. Когда четырнадцать лет назад правительство намеревалось проложить дорогу через Леале Имо – что было?
- Леале Имо – это Холм Предков на острове Воталеву? – уточнил Секар.
- Да. Тогда, как вы помните, памятники утафоа еще не охранялись правительством, да и с защитой личных прав утафоа были проблемы…

Секар улыбнулся:
- Еще бы я не помнил! Мой отец и старший брат стояли в живой цепи…
- И никто их к этому не принуждал, верно?
- Скорее уж наоборот. Мама опасалась, что будет драка с полицией…
- А мы в этой цепи познакомились, - Лайша толкнула Грендаля в бок, - помнишь?
- Ну, еще бы, - он подмигнул жене, - ты еще сказала «похоже, мы сейчас огребем».
- Ага! А ты ответил «спорим на пиво, что копы сдрейфят».
- Вот это интересно! – заявил репортер, - можно подробнее?

Лайша фыркнула.
- Да ничего особенного. Мы проторчали почти сутки нос к носу с копами. Они кричали в свой мегафон «вы оказываете незаконное сопротивление полиции! мы вынуждены будем применить силу!». А мы кричали в свой мегафон «прочтите свои контракты, пока не вылетели с работы! это ничейная земля, и мы будем тут стоять до решения суда!». К вечеру второго дня приехал судебный пристав с бумагами, копы сели в катера и свалили. Вот так я проиграла пари и поила этого типа пивом.
- Но закуска была за мой счет, - напомнил Грендаль.
- В кабаке – за твой, а у меня дома ты потом слопал все, что было в холодильнике.
- Ой, много ли там было? Тощая курица и кусочек сыра.
- А яичница из четырех яиц на завтрак?
- Ну… я посчитал их вместе с курицей, для краткости. И вообще дело прошлое.
- А помнишь, как ты нашел эти пакгаузы?
- Я помню, что ты назвала их гробиками для динозавров.
- Какие пакгаузы? – поинтересовался репортер.

Лайша рассмеялась
- Вы не заметили? Дом построен вокруг пакгауза. Соседние дома – тоже. На многих атоллах были военные базы и склады, а после революции всех иностранных военных отсюда выгнали. Те, конечно, забрали с собой все, что могли. Только голые бетонные коробки остались, и правительство стало их понемногу распродавать. А мы с Греном как раз решили жить вместе, и искали жилье подешевле. С деньгами у нас было, так сказать…
- Так и сказать: не было денег, - перебил Грендаль, - и тут я нашел объявление про эти пакгаузы. Отдавали их по 2000 фунтов, можно сказать, даром.
- Они и того не стоили, - заметила Лайша, - Четыре стены с дырками и без крыши.
- Крышу я сделал уже через неделю, - напомнил он.
- Ага, «сделал». Знаете, что он сделал? Подбил двух соседей, Ван Мина и Рохан Виджая, они тогда были такие же балбесы, как и он, и давай шакалить по окрестностям. Нашли разбитый самолет времен второй мировой, отволокли трактором на берег, раздраконили на части и поделили. Так что вместо крыши у нас было полкрыла и кусок фюзеляжа. Типа, мансарда с балконом. И трап вместо лестницы.
- Ладно тебе, нормально ведь получилось, - возразил Грендаль.
- Ну, да. Правда, первым же штормом нас чуть не сдуло оттуда в океан, а так нормально.
- Чуть не считается. А какой ветряк я сделал из пропеллера, помнишь?
- Еще бы! Иногда он так жужжал, что рыба в лагуне пугалась.
- Зато мы экономили на горючем для генератора. И вообще, разве плохо было?
- С тобой хорошо, Грен, - просто сказала она, - и тогда было хорошо, и сейчас.

- А почему вы мне про это не рассказывали? – обиженно спросил Иржи.
- А потому, что ты не спрашивал, - Лайша улыбнулась, - и, кстати, вот теперь тебе точно пора мыться и спать.
- Сейчас. Только дойду до 9 уровня и…
- Десять минут, договорились? – перебила она.
- Пятнадцать.
- Ладно, но ни минуты больше… Сен Секар, а вы это тоже предполагаете публиковать? Я имею в виду, то, про что мы сейчас говорили.
- Ну, вообще-то… - репортер замялся, - … Мне кажется, и ваше участие в защите Холма Предков и история вашей жизни здесь, с соседями разного этнического происхождения и, наверное, разной религии, так?
- Ну, разной, - согласилась она, - подумаешь, большое дело.

Секар энергично закивал.
- О том и речь. Это очень важная деталь. Так что, если вы не сильно возражаете.
- Я-то не возражаю, - Лайша пожала плечами, - чего тут такого.
- Я тоже не возражаю, - сказал Грендаль, - хотя не понял, почему это важно.
- Важно вот почему. На обвинение в нетерпимости к чужим взглядам вы, сен Влков, ответили спикеру европейской комиссии: «ваша толерантность – это просто трусость». Ваши слова были истолкованы, как апология жесткой идеологической унификации.
- Скажите уж прямо: фашизма.
- В общем, да. А после всех ваших историй, об этом даже говорить смешно.
- Ладно, вы – пресса, вам лучше знать.

Репортер улыбнулся и снова кивнул.
- Для уточнения вашей позиции я задам еще вопрос: рассказывая о свином буме, вы упомянули, что отец ребенка недостаточно тактично изложил свои претензии. А что это значит, и как он мог бы сделать это тактично?
- Он сказал примерно так: ислам учит, что свинья – нечистое животное с этим следует считаться, вы не вправе оскорблять мою веру. Он стал диктовать свободным людям, на что они имеют право, а на что – нет. Если бы он сказал: сын очень страдает из-за этого поросенка, и, если эта картинка для вас не принципиальна, то нельзя ли попросить ваших детей писать ручками с другой картинкой – реакция, наверное, была бы другой.
- Милосердие? – спросил репортер.

- Вроде того, - Грендаль пожал плечами, - В начале-то никто и не думал терроризировать мальчика этими поросятами. Моральный террор начался только в ответ на попытку принуждения. Когда к нам в гости заходит одна милая дама, вегетарианка, мы не ставим на стол мясо. Это не из уважения к вегетарианскому учению, а просто чтобы не обидеть человека из-за ерунды.
- То есть, - сказал Секар, - если бы вегетарианцы потребовали прекратить употребление мясной пищи в общественных местах…
- … То я бы демонстративно жрал сосиски в центральном парке, - закончил Грендаль.
- А если бы они не потребовали, а попросили?
- Тогда я бы не обратил на это внимания. Каждый вправе агитировать за что хочет, в пределах допустимого Великой Хартией, но эта агитация не вызывает у меня отклика.
- Иначе говоря, вы готовы пойти на уступки обременительным для вас странностям индивида, но не общественной группы?
- Верно. Потому что каждому индивиду свойственны какие-нибудь странности, но в общественной деятельности они неуместны.
- Но в случае с Холмом Предков вы, тем не менее, пошли на уступки странностям религии аборигенов.

Грендаль сделал энергичное движение ладонью, будто отталкивал препятствие.
- Ничего подобного, сен Секар. Мы встали в живую цепь, чтобы защитить объективные права людей, которые по объективным же причинам не могли сделать это сами. Право на сохранение своих святилищ есть у каждого, какие тут странности? Религия ину-а-тано и ее святилище Леале Имо - не исключение. Великая Хартия одна для всех.
- А если бы правительство решило проложить шоссе на месте мусульманской мечети, вы, сен Влков, встали бы в живую цепь, как тогда?
- Нет. Но если бы мне, как судье, подали жалобу, я запретил бы разрушать мечеть.
- Уверен, так бы и было, - сказал Секар, - но вы не стали бы лично защищать святилище ислама, как защищали святилище ину-а-тано. Вы не считаете эти религии равными?
- Не считаю, - подтвердил Грендаль.
- А как же Хартия?
- При чем тут Хартия? Хартия требует прямых действий гражданина в трех случаях: если человек в опасности, если попирается правосудие и если узурпируется власть. Ошибочное разрушение чьих-то святилищ сюда не относится. Гражданин может вмешаться в такую ситуацию на свой риск, но он вовсе не обязан этого делать.
- Но разве Хартия не обязывает нас считать все религии равными?
- Нет. Она лишь говорит о равных религиозных правах. Каждый может практиковать любую религию, и никто не вправе мешать ему, если эта практика не нарушает ничьих прав. Но каждый может проявлять симпатию к одним религиям, и отвращение – к другим. Поэтому во время «свиного бума», суд постановил изъять плакаты «мусульмане, вон из страны», но не трогать плакаты «ислам – дерьмо, мусульмане - свиньи».
- Все равно это жестоко. Большинство мусульман не участвовали в беспорядках. Их-то за что так?

- Понимаю, им обидно, - задумчиво сказал Грендаль, - Мне кажется, их проблема в том, что они не осудили своих радикалов. Поступи они так, как наши индуисты в казусе со шлягером «аватара Кришны» или как наши католики в истории с папской энцикликой «о сатанинской природе евгеники» - проблем бы не было.
- Но наших католиков за это отлучили от церкви, - напомнил Секар, - не думаю, что им было приятно.
- Да, наверное, - согласился Грендаль, - но тут приходится делать выбор: быть гражданами или слугами церковного начальства. По-моему, они сделали правильный выбор. Теперь у них своя католическая церковь, со статутами утвержденными постановлением Верховного суда, и я не замечал, чтобы наши католики очень страдали от такого положения.
- Ну, не знаю, - возразил репортер, - Ведь Ватикан и Всемирный совет церквей не признали это постановление и добились резолюции Объединенных Наций о произволе с церковным имуществом.
- Подумаешь, ООН. За 20 лет эти клоуны не выполнили ни одной своей резолюции.
- Я могу это привести эти ваши слова в репортаже, сен Влков?
- Конечно, а чего церемониться? Пока в ООН имеют право голоса торговцы кокаином, сексуальные маньяки, фанатики, террористы и людоеды, она не может претендовать на международный авторитет. Я так прямо и сказал их эмиссару.
- Представляю, что там было, - заметил Секар, шлепая по клавиатуре, - а вы знаете, председатель Всемирного совета церквей назвал Великую Хартию «новой опасной и агрессивной религией».
- Что, правда? – спросил Грендаль, - хотя, я не удивлен. Когда огласили постановление о депортации их миссии, их представитель кричал, что Конфедерация во власти сатанистов. Сатанисты – это, кажется, тоже религия. Вы не в курсе?
- Не знаю, сен Влков. Наверное, да, ведь про сатану вроде бы написано в библии.
- Вот и я не знаю… Сен Секар, это конечно ваше дело, но вы не опоздаете на встречу с Джеллой? У вас мощная машина, спора нет, но до Акорера почти тысяча километров.
- Уф! Постараюсь не опоздать. У меня еще последний вопрос: вы сами религиозны?
- Я? Ну, мне кажется, что-то такое есть, но чем оно может быть – понятия не имею.
- Можно так и записать?
- Можно. Почему бы и нет.

6. Джелла Аргенти, верховный судья по рейтингу.

После атолла Сонфао, остров Акорера казался огромным, хотя был всего 80 километров в длину и около 30 в ширину. Клуб рок-спорта, построенный по проекту гениального Хен Туана несколько лет назад, располагался на узкой северной оконечности острова. Две готические башни, поднимающиеся, казалось, прямо из океана, а посредине - наполовину встроенная в склон скалы трехъярусная пагода из стекла и бетона. Композиция должна была символизировать постмодернистский синтез культур Запада и Востока, но местная молодежь по какой-то причине дала зданиям клуба имена из «Одиссеи» Гомера: пагода стала называться Итакой, а башни - Сциллой и Харибдой. Малик, позвонил Джелле с вопросом, где ее искать, и услышал в ответ: «верхний ярус Итаки со стороны Харибды».

Середину яруса занимал огромный цилиндрический аквариум с яркими рыбками, со стороны скалы к нему примыкал бар со стойками в обе стороны, а в остальной части помещения было что-то похожее на материализованные фантазии Сальвадора Дали – искривленные причудливым образом ажурные конструкции, служившие сидениями и столиками. Малик прошел на левую, ближайшую к Харибде, половину и начал шарить взглядам среди посетителей. Публика, одетая в разнообразные модели легких, ярких спортивных или купальных костюмов, или просто обернутая в куски ткани, располагалась на разной высоте, подобно стае экзотических птиц на ветвях какого-нибудь баобаба.

- Эй, бро, ищешь кого-то? – флегматично поинтересовался бармен, не выпуская изо рта дымящуюся яванскую сигару.
- Да, Джеллу Аргенти.
- Ну, тогда ты пришел вовремя. Она только что доиграла гейм, но еще никого не склеила. Пол-оборота налево. Видишь четверть попы в лиловом платочке наискось?
- А! - со значением произнес репортер.
- Ага! - согласился бармен.

Пройдя до середины чего-то вроде кривого мостика и поднявшись на полвитка винтовой лесенки, Малик оказался лицом к лицу с объектом своих поисков. Крепкая невысокая девушка лет 27, одетая только в прямоугольник тонкой ткани, пропущенный под левой подмышкой и закрепленный над правым плечом фибулой в виде темно-красного спрута.
На открытых левом плече и правом бедре красовались два значка ронго-ронго: «стрела» и «рыба» соответственно, нанесенные ярко-зеленой люминесцирующей краской. Портрет дополняли темные прямые жесткие волосы, широкие скулы, маленький вздернутый носик и огромные почти черные глаза. В общем, смотрелась Джелла Аргенти эффектно.

- Значит, так, - сказала она, - если нет возражений, то на ты и по имени. Без этих церемоний. Йо?
- Йо, - согласился он.
- Тогда падай за столик и включай свою машинку. Ты саке пьешь?
- Так, чтоб поддержать компанию, - ответил он.

Она взяла керамический кувшин и плеснула остро пахнущий напиток в две чашечки.
- Ну, давай, стартуй.
- Э… - Малик пригубил саке, - ты вообще кто по профессии?
- Конфликтолог. Работаю в морской авиации. Там бывают такие корки, что этот суд для меня, считай, каникулы.
- А как ты прошла в конкурсную тройку профессионалов Верховного суда?
- Да обыкновенно. По рейтингу выступлений. Фишка в том, что я умею говорить просто о сложном. На флоте без этого никак. Сечешь?
- Помаленьку, - ответил Секар, - а ты можешь просто объяснить, как вы принимали это решение?
- Постановление о депортации? – уточнила она, - да не фиг делать. Ты в политике рубишь, или как?
- Наверное, не на том уровне, чтобы…
- Ясно, - перебила Джелла, - тогда погнали от ворот. Рисуем такую схемку…

Она быстро набросала на салфетке несколько квадратиков, кружочков и стрелочек и начала комментировать:
- Вот этот кружочек – любой гражданин. Он работает по найму или на свой бизнес, без разницы. Чего-то наживает и чего-то покупает для себя. Кроме жратвы, хаты, тачки он еще покупает общественный порядок. Порядок – это такой же товар, сечешь?
- Порядком занимается правительство, - заметил репортер.
- Точно! И фишка в том, что оно - естественный монополист. Ведь порядок должен быть один для всех, так? А значит - что?
- Значит, правительство должно быть одно, - ляпнул Секар.

Джелла махнула рукой.
- Это ясно. Но главное – оно должно продавать то, что гражданину нужно, а не всякую лишнюю фигню, и цена должна быть справедливая, а не заряженная. Врубаешься?
- Да.
- …А значит, приходим к заявкам, запросу и конкурсу, - продолжала она, - форма избирательной заявки определена в Хартии. Ты избирательные заявки заполнял?
- Конечно.
- Вот. Заявки граждан усредняются и получается карта общественного запроса. Ничего сверх этого запроса, правительство делать не имеет права.
- Ну, это я, положим, знаю, - обижено заметил Секар.
- Йо! Дальше – конкурс команд-претендентов. Координаторы - раз, фонды - два, армия - три, полиция - четыре, преторианцы - пять. По каждому из пяти, какая команда обязалась дешевле удовлетворит запрос – с той и заключается генеральный контракт. Команда координаторов - это правительство. Оно имеет право собирать с граждан равные взносы, чтобы в сумме получалась цена всех пяти контрактов.
- Это я тоже знаю.
- Дальше – суды трех уровней: муниципалитет, округ, конфедерация. По шесть человек, трое непрофессионалов по жребию, трое профессионалов по общественному рейтингу. Над Верховным судом конфедерации - только Великая Хартия, а если кто-то этого не понял – преторианцы вправят ему мозги. Если кто-то снаружи хочет навязать другой порядок – армия должна гасить его безо всяких правил. Въезжаешь, почему?
- Потому, что на войне вообще с правилами сложно, - предположил Секар.
- Потому, что без правил дешевле, - поправила Джелла, - хотя по жизни ты прав. Какие, на фиг, правила на войне. Вот, вчерне, и вся политика. Йо?
- Ну, да. Примерно так я себе и представлял. Ничего особо сложного, правильно?
- Верно, Малик. Согласно Хартии, политическая система должна быть такой простой, чтобы ее понимал каждый житель со средним образованием. Иначе жители не смогут осмысленно реализовывать управление страной.

Секар отхлебнул саке и спросил:
- Джелла, а какое отношение все это имеет к постановлению о депортации?
- Прямое, бро. Эти типы требовали, чтобы правительство делало то, на что не имеет права.
- То есть, - уточнил он, - мы приходим к разнице между правительством и государством?
- Вот именно.
- А можешь максимально кратко напомнить эту разницу?
- Правительство обслуживает людей, а государство управляет ими.
- Пожалуй, это слишком кратко.

- То-то же, - Джелла усмехнулась, - ладно, объясняю на пальцах. Когда ты заказываешь уборку дома, тебя интересует, чтобы за определенную плату там навели чистоту и тебе по фигу, кто конкретно это сделает. Теперь прикинь, если ты заказал, кто будет делать, но не определил, что именно делать и почем. Ты вернулся домой и видишь: чистота так себе, зато книги на полках и картинки на стенах другие, чем были, ящик стола вскрыт, часть писем выброшена, а вместо халата в ванной висит пижама в клеточку. Стоимость всех этих художеств включена в счет и внизу приписка: мы решили, что так будет лучше.
- Это почему еще?
- Это потому, что у тебя похозяйничало государство. Государство – это каста, которая предписывает обществу какие угодно законы и взимает с людей какие угодно подати. Восточная деспотия делает это открыто, а западная демократия это маскирует с помощью выборов, но суть одна и та же. Государство может заставить тебя отчитываться обо всех доходах и платить в бюджет любую долю от них. Государство может навязать тебе такие правила бизнеса, что ты останешься нищим. Государство может оштрафовать тебя и твою подружку за то, что вы пьете вино и спите вместе без специального разрешения.
- То есть, государство может делать с людьми вообще что угодно?
- Йо! Собирается человек 500 из этой касты, штампуют специальный закон – и все.
- Но есть же выборы. Почему не избрать вместо этой касты других людей?

Джелла ответила характерным жестом, хлопнув ладонью левой руки по сгибу локтя правой, после чего пояснила:
- У тебя ни черта не выйдет. Каста пронизывает все структуры управления и все каналы массовой информации. Выборы устроены так, что шансы есть только у членов касты. Я лично знаю только один проверенный способ это изменить.
- Алюминиевая революция?
- Она самая.
- То есть, - сказал репортер, - смысл алюминиевой революции был в том, чтобы никто за людей не решал, как для них будет лучше?
- Ага. А кто пробует решать - тому расстрел или депортация, смотря по обстоятельствам.
- Понял. Кажется, мы добрались до сути дела, а?
- Йо, - Джелла энергично кивнула.
- В таком случае, помоги разрешить одну дилемму о правах граждан. Граждане ведь могут прибегнуть к уличным акциям, если нарушены их права?
- Запросто, - подтвердила она.
- Вот, - продолжал Секар, - группа граждан выходит на улицу с требованием прекратить их дискриминацию по религиозным и моральным убеждениям. Что здесь неправильно?
- Уточни их требования. Что написано на транспарантах?
- Кажется так: Прекратить унижение веры. Долой культ разврата.
- Ну и при чем тут дискриминация? - спросила Джелла, - если им не нравится, как кто-то отзывается об их вере, то это их проблемы, а разврат вообще безразличен для Хартии.
- Но в их заявлении пояснялось, что они подвергаются унижению, как социальная группа.
- Бред, - отрезала она, - объектом дискриминации могут быть только конкретные люди. Никто из этих типов не был лично ограничен в правах по сравнению с другими людьми.
- Это точно? – спросил репортер.
- Абсолютно. Ни одна социальная анкета даже не содержит графы «религия». Это такой же приватный вопрос, как пищеварение.
- Кстати о пищеварении, - сказал он, - как быть, например, со школьными занятиями?
- Ты о чем?
- Об уроках биологии человека. В ряде религий это считается неприемлемым.

Джелла презрительно фыркнула.
- Бро, этот вопрос разъяснен 8 лет назад в деле Оскар. Согласно Хартии, школа служит, чтобы давать молодежи актуальные навыки и знания о природе, человеке и обществе. Для этого нужно показать свойства человеческого тела. Если в какой-то религии табу на это…
- То что делать представителям такой религии? – перебил Секар.
- Это - их проблемы. Может, у кого-то таблица умножения считается непристойной.
- Но, согласись, это означает религиозную дискриминацию.
- Нет. Если у человека в аттестате прочерки, он в худших условиях не из-за религии, а из-за отсутствия знаний. Семья Оскар ссылалась на практику стран, где не преподают то, что считается недопустимым в их религии, но суд разъяснил, что это противоречит Хартии.
- Почему?
- Потому, - сказала Джелла, - что они требовали не увеличения своих прав, а уменьшения прав остальных школьников. Они хотели не получить что-то себе, а только отнять что-то у других. Заведомо деструктивное требование. Врубаешься?

Секар почесал в затылке.
- Не уверен. А можно обратный пример на ту же тему?
- Легко. Китайцы и школьные бассейны. Когда мы подписали с Китаем договор о дружбе, в Меганезию приехало полмиллиона семей. Вдруг сюрприз: большинство их детей не умеют плавать, а ведь здесь океан для детей - это… Ну, понимаешь.
- Еще бы! – согласился репортер, - ни один школьный пикник без этого не обходится.
- Китайцы учредили родительские комитеты и забросали всех жалобами, - продолжала она, - почему на физкультуре не учат плавать? Соблюдайте Хартию! Раньше это никому в голову не приходило, обычно здесь дети учатся плавать раньше, чем ходить, а тут - факт налицо. Плавание - актуальный навык, и школа обязана этому учить.
- Выходит, эти бассейны появились из-за китайских иммигрантов?
- Выходит, так.
- То есть, - резюмировал Секар, - они требовали что-то для себя, и это конструктивно?
- Йо!
- Понял. Теперь давай я расскажу по-своему, а ты поправишь.
- Валяй, - согласилась она, отхлебывая саке.



7. Порядок - для человека, а не человек - для порядка


Репортер последовал ее примеру, после чего выдал:
- Приходит сын с митинга коммунистов. Отец спрашивает: чего они хотят? Сын отвечает: чтоб не было богатых. Отец удивился: а почему они не хотят, чтоб не было бедных?
- Ну… - задумчиво протянула Джелла, - Да, вроде того. Идефикс о порядке ради порядка.
- Этого я уже не понял, - признался он.
- Это элементарно, Малик! Взять те же школьные бассейны. Комитет «Наша семья» потребовал не пускать туда на переменах. Подросткам не охота возиться с мокрыми тряпками и многие купаются голыми. Комитет утверждал, что это аморально, а права человека должны быть ограничены справедливыми требованиями морали. Суд ответил, что если и есть такое требование морали, то оно несправедливо.
- А как определили справедливо или нет?
- Суд взял тезис Платона: справедливость – это такой порядок, при котором каждый человек в полной мере реализует данные ему от природы способности.
- Порядок - для человека, а не человек - для порядка? – уточнил Секар.

Джелла хлопнула в ладоши в знак одобрения.
- Йо! Любая другая трактовка противоречила бы Хартии.
- Ясно, - кивнул Секар, - Еще вопрос. Ты ведь ведешь дело о халатности полиции?
- Да, и что?
- Я, конечно, понимаю, что до решения суда…
- Фигня, - перебила она, - свое личное мнение я могу сообщать прессе. Готов?
- Конечно. Я весь внимание.
- Значит, так. В идеале беспорядки должны пресекаться мгновенно, но идеал – это более дорогое удовольствие, чем фонды полиции. Полиция должна устранять обычные угрозы для граждан и сообщать о необычных фактах социальной напряженности. Она сообщала о риске беспорядков вокруг «Детей троглодитов». Все могли принять меры, но этого не сделал никто. 7 объектов были разгромлены за 10 минут, а потом группа экстремального реагирования пресекла погромы. 10 минут для них нормативное время по контракту.
- Но на площади Ганди, где шел митинг против «Детей троглодитов» находились трое полицейских, - заметил репортер, - они обязаны были…
- Что обязаны? – перебила Джелла, - стрелять по толпе? А результат представляешь?
- Но «экстремалы» же открыли огонь сразу.
- На то они и экстремалы. Их учат, как и в кого стрелять при массовых беспорядках. Как ты думаешь, почему все обошлось минимальными жертвами?

Секар задумался на несколько секунд и спросил:
- Ты хочешь сказать, что если бы те трое полицейских открыли огонь…
- Была бы мясорубка, - снова перебила Джелла, - одна неприцельная очередь из автомата по плотной толпе это несколько убитых или тяжело раненых. А была бы не одна.
- То есть, полиция действовала правильно?
- Скажем так, удовлетворительно. Конечно, задним числом можно много чего придумать, но никто не ожидал, что эти психи решатся на погромы. Ведь во время «свиного бума» всем дали понять: тут не Европа, тут с фанатиками не церемонятся.
- А что будет с убытками? – спросил Секар.
- Вероятно, мы, взыщем с полиции стоимость поврежденного имущества. В конце концов, у них на это есть страховка. Но я буду против взыскания упущенной выгоды от потери клиентов. Скандалы действуют как реклама. Клубы уже увеличили свою клиентуру.
- Логика понятна, - сказал он, - и еще вопрос о депортации. Джелла, а по какому принципу были высланы именно эти 19 человек? Мировое общественное мнение считает, что имели место репрессии по идеологическим мотивам.
- Это чушь, бро. Они подстрекали против общественной безопасности и Хартии.
- Каким образом?
- Так, как это обычно делается. Например, пастор, который кричал в мегафон...
- Джереми Вудброк, - подсказал он.
- Да, Вудброк. На видеозаписи есть, как он призывает поджигать и громить. После этого был разбит стеклянный фасад кинотеатра, а внутрь брошены бутыли с бензином.
- Он утверждает, что просто читал из библии, - заметил Секар, - и я проверял, это - правда. Глава 7 книги Второзаконие: «поступите с ними так: жертвенники их разрушьте, столбы их сокрушите, и рощи их вырубите, и истуканов их сожгите огнем».

Джелла презрительно фыркнула:
- Свинья грязи найдет. Хоть в библии, хоть в букваре. Суду плевать, откуда он читал.
- Но для кого-то библия – священная книга, в которой верно каждое слово.
- Эти их проблемы.
- Это их право, - возразил репортер, - свобода религии есть в Хартии.
- Свобода религии не означает свободу творить на улице все, что написано в какой-нибудь священной книге, - отрезала она, - чувствуешь разницу?
- Это относится к Вудброку, - сказал Секар, - но другие представители Всемирного совета церквей в уличных беспорядках не замечены. А правозащитники из Комитета-48…

- Понятно, - перебила Джелла, - сейчас…

Она наклонилась, вытащила из-под столика спортивную сумку и стала в ней рыться. Некоторое время мелькали разные предметы, как-то: теннисная ракетка, форменное кепи ВВС Меганезии, журнал «подводная охота», мобильный телефон, маска для дайвинга… Наконец в ее руках оказался электронный блокнот.

- Вот, нашла! И почему у меня вечно такой бардак?
- Говорят, беспорядок в сумочке – признак женственности, - ляпнул Секар.
- Да? Ну, тогда не обидно. Окей, начнем с Всемирного совета церквей. Они издали заявление «Вера и Право», где дословно говорится: «Так называемая Великая Хартия защищает право на грех, а грех не должен иметь защиты, с ним нужно бороться и искоренять его. Свидетельство веры требует дел. Общество должно быть очищено от таких законов, которые оправдывают безнравственность, отдавая веру и мораль на поругание». Дальше – подписи. Это публичный призыв к уничтожению Хартии, такое карается высшей мерой гуманитарной самозащиты.
- Они говорят, что их репрессировали за веру, - вставил репортер, - и ссылаются на опыт других стран, где их не преследуют за критику морального релятивизма.
- Нет проблем, - спокойно ответила Джелла, - мы их и выслали в другие страны. Теперь о правозащитниках. Здесь сложнее. Больной вопрос о семейных правах.

Секар кивнул, не отрываясь от клавиатуры.
- Сен Влков уже говорил мне. Ограничение прав семьи на выбор воспитания детей?
- Йо! - сказала она, - если на пальцах: конфликт прав ребенка с правами родителей. Родители хотят воспитать его по таким-то традициям, но тогда он окажется в жопе, потому что современное общество устроено не по традициям.
- Это очевидно, - согласился репортер.
- Не очень-то. Правозащитный Комитет-48 давил на то, что правительство обязано искать компромисс. А в Хартии сказано, что это, во-первых, не в компетенции правительства, а во-вторых, это вообще не компромиссный вопрос, права ребенка приоритетны.
- В Хартии так написано?
- Там написано: «Любой человек с момента рождения находится под безусловной защитой правительства, обеспечивающего базисные права каждого жителя страны».
- Но у родителей тоже есть права, - заметил Секар, - это ведь их ребенок.
- В Хартии сказано: «ни один человек не имеет никаких прав на другого человека, кроме случая принудительных гражданских ограничений и компенсаций».
- То есть, ты хочешь сказать, что мой ребенок – это как бы и не мой ребенок?
- Твой. Но не в том смысле, как твоя табуретка. Своей табуретке ты вправе отпилить ножку, а своему ребенку…
- Бррр… Джелла, ну у тебя и примеры, однако…
- Это для доходчивости, - пояснила она, - воспитание в традициях, скажем, пуританства или парсизма – это увечье. Оно объективно лишает человека возможности нормально общаться со сверстниками, получить полноценное образование, участвовать в социально-культурной жизни, найти достойную работу. Дети – не собственность родителей, а люди. Они под защитой правительства. Правительство обязано вмешаться в дела семьи, если это необходимо для защиты прав личности, так говорит Хартия.
- Но… - попытался вставить репортер.

Джелла остановила его предостерегающим жестом.
- Не перебивай, Малик. Да, правительство действует жестко, зато у нас практически нет насилия в семье. Гуманитарные организации даже подозревали нас в обмане и собирали независимую статистику. Потом признали: да, здесь мы опережаем весь цивилизованный мир с огромным отрывом. Дальше нас заподозрили в чрезмерном давления правительства на семью, но оказалось, что и этого у нас намного меньше, чем в других странах. Наконец, нас обвинили в тотальном подавлении культурных общин. В ответ координатор Накамура опубликовал коммюнике правительства, из которого я зачитаю кусочек.

Она потыкала в свой электронный блокнот:
- Ага, это: «Хартия признает субъектом прав только человека. Если какая-то группа людей желает заявить о своих коллективных правах - она создает корпорацию, представляющую лишь тех, кто в нее вступил, и лишь по вопросам, которые он ей делегировал. Этническая или религиозная принадлежность не есть принадлежность к корпорации. Это значит, что никто не может заявлять о правах этноса или религии и выступать от имени всех лиц, к ним принадлежащих. Заявления такого рода будут игнорироваться правительством». Все. Верховный суд признал коммюнике соответствующим Хартии и контракту правительства.

Секар покачал головой:
- Вот уж действительно жестко.
- Йо! - согласилась Джелла, - но подход себя оправдал. Правительство открыто наплевало на требования, исходящие, якобы, от всех индусов, всех христиан или всех европейцев, и оказалось, что так называемые «все» - это кучка политических аферистов. Их взгляды не разделяются большинством культурной общины. Все здорово упростилось. Взять хотя бы случай с папской энцикликой о евгенике.
- Да, сен Влков упоминал об этом.

Джелла кивнула и продолжала:
- Потом от иллюзий про «всех» избавилось и большинство индивидов, принадлежащих к культурным общинам. Есть исследования поля мнений. Подростки все чаще говорят об общей меганезийской культуре, в которой есть вклад европейцев и африканцев, китайцев и индусов, всей уймы этносов, культур и религий, которые тут перемешались за 200 лет.
- Да, наверное, - согласился Секар, - когда я ляпнул про отдельную культуру аборигенов, младший Влков глянул на меня, как на дебила, и обругал на языке утафоа.
- А чего ты ждал? - спросила Джелла, - еще скажи, что сонеты Шекспира это отдельная культура британцев.
- Ты меня запутала, - сказал он, - то говоришь, что культура у нас не защищена вообще, то наоборот, что она защищена лучше, чем где-либо.
- Да какая, ерш ей в дюзы, защита! – взорвалась она, - культура это жизнь общества, она неотделима от общества. Пока общество живо, с культурой ничего не может случиться! Попробуй, тронь культуру - общество тут же снесет тебе башню.
- Зачем тогда придумали акты о защите культурных прав? – спросил репортер.
- Затем, что некоторые государства недовольны той культурой, которую общество создает и потребляет естественным путем. Ты посмотри, что защищается под видом культуры! Не Гомер, не Шекспир, и даже не Микки Маус.
- А действительно, что защищается?

- Вот это правильный вопрос, бро, - одобрила Джелла, - защищается то, что обществу на фиг не нужно, зато нужно типам, которые говорят «за всех». Мы эту проблему решили жестко, а западные политики спасовали перед кучкой аферистов и психически увечным отребьем. Струсили и пытаются выкрутиться через толерантность. Мол, давайте будем делать вид, что не замечаем их психических увечий. Во избежание конфликтов, будем во всем потакать этим уродам. Будем избегать того, что может их обидеть. Неизбежный результат: нормальным людям приходится вести себя так, будто они тоже изувечены. Толерантное общество строится под уродов. Норма объявляется увечьем, а уродство - социальной нормой. Знаешь, бро, в чем причина скандала вокруг «детей троглодитов»?
- Не уверен. Скажи лучше сама.
- Ладно, скажу. Там, - Джелла махнула рукой на закат, - уроды привыкли, что в гуманном постиндустриальном обществе все под них строятся. Ни один сраный фундаменталист не стал бы так выпендриваться во Вьетнаме. Там марксистская индустриальная технократия, там за это… - она прицелилась указательным пальцем в лоб собеседнику, - пиф-паф и все. А у нас они рассчитывали всех построить под себя. Размечтались…
- А при чем тут Комитет-48? - спросил он.
- При том. Они напечатали отчет: в Хартии 16 противоречий с актами ООН о семейных и культурных правах, и предложили Генеральной ассамблее проект экономических санкций против Меганезии до ликвидации этих противоречий. Не будь проекта – их не привлекли бы к суду, у нас свобода слова. А тут - публичный призыв к уничтожению Хартии.
- И что, этот проект может пройти?

Джелла задумчиво подвигала чашечку по столу.
- Черт его знает, я тут не спец. Но, по-моему, у них пороху не хватит.
- Понятно. А на несколько вопросов о себе можешь ответить.
- Легко. Что интересует?
- В общих чертах - семья, хобби, религия.
- Смотря что называть семьей. Как минимум, это я и мой трехлетний сын. Но, поскольку я девушка мобильная, он много времени проводит у мамы и ее третьего мужа, либо у папы и его второй жены, либо у моего экс-бойфренда, его технического папы. Правда Энди (это парень с которым я в основном живу), предпочитает, чтобы мы сами больше занимались сыном. Он в чем-то прав, ведь если мы заведем еще ребенка (а почему бы нет?), то опыт…
- Стоп, стоп, - Секар беспомощно поднял руки, - я запутался.
- Ничего удивительного, я сама иногда путаюсь.
- Гм… Можно я напишу так: живет в большой семье, воспитывает сына?
- Нормально, - согласилась она, - что там еще? Хобби – дайвинг. Религия – католицизм.
- Католицизм? – удивился репортер, - ты верующая католичка?
- А что такого? В конце концов, почему бы там, - Джелла ткнула пальцем вверх, - не быть кому-нибудь, кто сотворил эту прикольную вселенную.
- Да нет, просто ты… Скажем, так, не очень похожа...
- Фигня. Католическая церковь учит, что ему, - она снова ткнула пальцем вверх, - это все равно. У него с чувством юмора все в порядке.
- Католическая церковь так учит? – переспросил он, - Никогда бы не подумал. Ах да, вы же отделились от Ватикана.
- Точнее, мы их выгнали отсюда на фиг. Наш консультант, доктор теологии из Оксфорда, научно доказал, что римские папы – самозванцы, и написал хороший понятный катехизис на 5000 знаков. Его удобно читать на мобильнике или элноте, - Джелла постучала ногтем по электронному блокноту, - им пользуются не только здесь, но и в Южной Америке, Индии и Австралии. На сайте нашего епископства можно скачать текст и аудиофайл.
- Непременно почитаю, - сказал репортер, - или послушаю.



8. Эрнандо Торрес, координатор правительства.


В редакции «Pacific social news» был привычный аврал, сопровождающий доводку утреннего номера. Шеф отдела политических новостей, пробурчав что-то вроде «тебя за смертью посылать», выхватил у Секара из рук флеш-карту и папку с бумажными копиями, после чего нырнул в лифт и унесся на этаж, где шла верстка.

- Ни тебе «привет», ни тебе «как дела», - буркнул Секар в пространство.

Часы показывали четверть пятого. Значит, Хелена уже давно спит без задних ног, и торопиться домой не имеет смысла. Придя к этому умозаключению, он решил зайти на часок в кафе, узнать свежие новости и поболтать с коллегами. Собственно, так делала почти вся горячая смена, так что в кафе уже болталось полдюжины человек. Услышав громкие хлопки, топанье и свист, Секар подумал было, что народ смотрит футбол. Оказалось – ничего подобного. В телевизоре наблюдался круглый стол под эмблемой ABC-online, и дело там, судя по жестикуляции участников, шло к точке кипения.

В кафе эмоции тоже били через край, и в центре бузы находилась Инаори Атаироа из отдела программного обеспечения. Одета она была по обыкновению в линялые джинсовые шорты и ослепительно-белую рубашку с короткими рукавами. Рубашка была расстегнута и завязана узлом примерно в районе пупа, так что можно было описать фигуру девушки практически полностью. Но только весь фокус был не в фигуре, а в той неуловимой пластике, которой отличаются утафоа (а что Инаори относилась именно к этой расе, было видно за километр). Как правило, вокруг нее увивались пять-шесть мужчин, но сейчас, под утро, их оставалось всего двое.

Один: вечно улыбающийся Эрнст Оквуд, инженер-электрик, умудрявшийся носить рабочий комбинезон так, будто это фрачная пара за 5 тысяч фунтов. Он переселился в Меганезию сравнительно недавно, по его собственным словам: «потому, что здесь гораздо веселее, чем в Глазго». Видимо, это и впрямь было так, поскольку последние два года на родине он провел в зарешеченном помещении из-за того, что поработал с сигнализацией одного банковского офиса тем способом, который по законам Соединенного Королевства называется «соучастие в краже со взломом». Попался он чисто случайно, так что теперь его квалификация ни у кого сомнений не вызывала.

Другой: рослый, атлетического сложения сикх (не путать с индусом, иначе он обидится), с обманчиво-мечтательным выражением лица. Звали его Лал Синг, и в прошлой жизни он был лейтенантом корпуса быстрого реагирования флота Меганезии. 2 года назад, во время операции в эмирате Эль-Шана, осколок мины, попавший в колено, поставил жирный крест на его армейской карьере. Он мог бы жить на страховую пенсию, но это ему было скучно. Он попробовал работать в полиции. Ему и там было скучно, и возможно, из-за этого, написанные им протоколы оказывались похожи на литературные миниатюры. Вскоре, с подачи коллег, он попытал счастья в журналистском конкурсе и попал на должность военного обозревателя «Pacific social news».

В данный момент эти трое создавали столько шума, что заглушали динамики, и понять смысл происходящего на экране было решительно невозможно.

Парочка за столиком в центре кафе, наоборот, внешнюю политику игнорировала из принципа. Они играли в стоклеточные шашки, считая это занятие гораздо более осмысленным, чем любой теледиспут. Во всем, кроме этого общего мнения, парочка являла собой предельный контраст. Викскьеф Энгварстром, репортер криминальной хроники, был типичный норвежец, светловолосый, сероглазый, метра под два ростом. Джой Ше из отдела новостей науки, наоборот была миниатюрная и с совершенно неопределенным этническим типом. Ее можно было с равным успехом принять за малайку, китаянку, испанку, латиноамериканку или уроженку Северной Африки. Знание пяти языков и природный талант к самому бесшабашному флирту, позволял ей втереться в доверие к кому угодно, а два высших образования – физика и филология – давали возможность эффективно распорядится практически любой полученной информацией.


За столиком в углу сидел Чжан Чжан – вот к нему-то Секар и направился, по опыту зная, что у этого китайского дядьки есть замечательное свойство: всегда слышать все, что происходит и толково излагать суть любого дела. Был он не вполне определенного возраста, где-то между 50 и 60. В его биографии вообще все было неопределенное. Он прибыл в Меганезию (точнее - еще в британскую Океанию) года за 2 до Алюминиевой революции, как говорят, по идейным соображениям, принял непосредственное участие в минной войне против колониальной администрации острова Тинтунг, а затем был старшиной наемников при подавлении путча батакских националистов. Впрочем, это все были слухи, а сам он на вопросы о своем прошлом отвечал многозначительными цитатами из Лао Цзы («Тот, кто много говорит – часто терпит неудачу» или «Кто знает – не говорит .кто говорит – не знает»). Сейчас он возглавлял отдел экологии, но советоваться к нему ходили со всех этажей и по всем вопросам – от курсов акций до средств лечения кошек.

- Шумят, - констатировал Чжан, улыбнулся и налил коллеге полчашки цветочного чая.
- Да уж… - согласился Секар, - а что там?
Китаец пожал плечами.
- Такая женщина.
Ясно, что он говорил об Атаироа, которая, как типичная молодая островитянка, могла с пол-оборота завести любое количество мужчин, оказавшихся в радиусе 7 футов от нее.
- Я имею в виду: что показывают? – уточнил Секар свой вопрос.
- Координатор Торрес в Монреале отбивается от стаи псов, - лаконично ответил Чжан.
- Давно?
- 27 минут приблизительно.
- Ах вот как… и сильно ли его покусали?
- Не очень. Они глупые. Мешают друг другу.
- А зачем тот нервный дедушка в канареечном галстуке трясет авторучкой?
- Это какой-то юрист из Сорбонны, - проинформировал китаец, - думаю, это у него такая манера публичных выступлений. Он доказывает, что существующая в Меганезии система равного социального землевладения возникла нецивилизованным путем.
- У него получается?
Чжан улыбнулся и отрицательно покачал головой
- Он споткнулся на вопросе об основаниях прав частной собственности на здешние земли. Стал выводить эти права из объявления трансмалайских островов владением британской короны и их передачи ост-индской компании, но забыл, что здесь уже тысячу лет жили утафоа. Торрес разъяснил, что алюминиевая революция восстановила законы утафоа о равном праве всех жителей на земельные угодья и промысловые воды.
- Судя по реакции профессора, в Сорбонне этому не учат, - заметил Секар.

Эрнандо Торресу, было около 50. Подвижный, смуглый, среднего роста с аккуратным брюшком, координатор был одет весьма неофициально. Свободные серые брюки, яркая пестрая рубашка-гавайка и завязанный на ковбойский манер шейный платок в виде меганезийского флага – черно-бело-желтый трилистник на лазурном поле. По мысли революционных символистов, это обозначало союз трех рас, населяющих атоллы, но меганезийцы, не будучи склонны к пафосу, именовали это просто «наш пропеллер».

Наметанным репортерским глазом Малик Секар тут же определил: Торрес не стремится к победе в диспуте, его вообще не интересуют оппоненты, он «работает на камеру», т.е. следит за тем, чтобы произвести впечатление на зрителей, используя оппонентов, как фон для своего выступления.

Тем временем, шумное трио несколько угомонилось, и стало слышно, как Торрес говорит:

«… несколько теряюсь. То ли мне отвечать тем, кто обвиняет правительство Меганезии в экономическом анархизме, то ли, наоборот, тем, кто утверждает, что мы замордовали предпринимателей тотальным регулированием и надругались над правом собственности. Может быть, ведущий мне подскажет, с чего начать…».

Ведущий поерзал в кресле, улыбнулся в 32 фарфоровых зуба и ответил: «Знаете, мистер Торрес, такое фарисейство не ново, его придумал еще Геббельс. То есть лавочники свободно торгуют пивом, а вот земля и недра принадлежат германской нации, которая превыше всего, то есть Рейху, потому что Рейх представляет нацию…».

«Минуточку, - перебил координатор, - в Меганезии нет никого рейха, а то, что вы назвали, у нас принадлежит нации конкретно и без посредников».

«Коммунисты в России тоже так говорили, - выкрикнул кто-то с места, - мол, у нас все народное, и даже любая кухарка может управлять государством. А на деле народ был нищ и бесправен, все принадлежало единственной разрешенной партии».

«Должен признаться, - Торрес развел руками, - что в Меганезии ни кухарка, ни кто-либо другой, не может управлять государством, и никакой партии ничего принадлежать не может. И государство и партии запрещены Великой Хартией. Так что земля, недра и акватория принадлежат гражданам непосредственно. Каждому жителю принадлежит равная доля, которой он может пользоваться сам или передать в пользование другому».

«Пустые слова! – крикнул тот же оратор, - как это непосредственно? Кухарка может продать мне кусок меганезийской акватории?»

«Продать не может, я уже это объяснял. А сдать в аренду на срок до 5 лет – пожалуйста. Для этого ей достаточно заключить с вами договор и направить копию в фонд экономики и природы. Так, я сдаю свою долю акватории фирме Snailbot, и имею хорошие дивиденды, Это некоторый риск потери доходов, но при хорошем раскладе я получаю на 25% больше, чем в фонде. Если вы предложите еще больше – я готов заключить договор с вами и…».

Конец фразы координатора потонул в гвалте снова расшалившегося трио. Секар подумал, что азарта у ребят не меньше, чем на футболе – по крайней мере, если оценивать азарт по уровню производимого шума. Когда они, наконец, угомонились, вопрос Торресу задавал некто, похожий на профессора из викторианской эпохи:

«… Развиваю свое собственное частное предприятие, - говорил он, - как вдруг ко мне приходят социальные наблюдатели и говорят: отдавай нам половину акций. Если это не грабеж – то что тогда грабеж?»

«Грабеж, - возразил Торрес, - это когда отбирают нечто, не давая взамен ничего. Так, налоги в западных странах - это грабеж. А когда ваши акции обменивают на паи любого инвестиционного фонда по вашему выбору – это антимонопольная политика. Общество принимает превентивные меры против экономического насилия со стороны частных лиц. Это есть в любой стране, просто в Меганезии эта политика реализуется честно и открыто. Имеете что-то возразить?».

«Конечно, имею! Еще бы! – ехидно сказал профессор, - антимонопольные органы лишь контролируют предпринимателя, чтобы он не создавал искусственного дефицита, монопольного завышения цен и не совершал тому подобных злоупотреблений. А ваши наблюдатели отбирают у людей собственность, принудительно меняя ее на что-то».

«Вы создаете словесную путаницу, - ответил координатор, - право собственности это возможность распоряжаться имуществом по своему выбору независимо от воли третьих лиц, любым способом, физически не опасным для окружающих. Если у меня за спиной стоит чиновник, который указывает мне, как я должен распоряжаться – то, значит, я уже не собственник, а болван в преферансе, моими картами ходит другой игрок».

«Надо же, - фыркнул его оппонент, - Что ж тогда крупные бизнесмены не возмущаются?»

«Элементарно, - Торрес улыбнулся, - чиновника ведь можно подкупить, и тогда болваном окажется уже все общество. Именно так и происходит в большинстве развитых стран».

«Вы думаете, у нас не борются с коррупцией?» - возмутился профессор.

«Я не думаю, я знаю. У вас не запрещено лоббирование. Крупные компании не просто так делают огромные пожертвования в кассы политических партий. Инвестиции во власть - это очень выгодное вложение денег. Как говорят русские «кто девушку ужинает, тот ее и танцует». Вашу якобы демократическую власть танцуют спонсоры ваших политиков».

«А в Меганезии, хотите сказать, кандидаты в парламент оплачивают избирательную компанию из своего кармана?»

Торрес снова улыбнулся: «Вы просто не в курсе. В Меганезии нет парламента»

«Нет парламента? То есть, как нет?»

«Никак нет», - любезно пояснил координатор.



9. Короткий диспут о политэкономии.


Инаори, Эрнст и Лал хором заржали, так что начало выступления следующего оратора – вальяжного господина в дорогом костюме - было невозможно расслышать.

«… несправедливость. С простого рабочего сдирают столько же налогов, сколько с миллионера».

«В Меганезии нет налогов, - мягко напомнил Торрес, - есть взносы на производство общественных благ. Они зависят не от того, сколько у человека денег, а от того, сколько он потребляет этих благ. Обычно миллионер платит много больше, чем рабочий, но не потому, что у него больше доход, а потому, что у него больше объектов, обслуживаемых полицией, экологической службой, службой чрезвычайных ситуаций и т.д.»

Вальяжный господин погрозил пальцем: «Не заговаривайте нам зубы! У вас живет Хен Туан, один из самых модных архитекторов мира. Он получает 4 миллиона фунтов в год, а его семья платит налог всего 10 тысяч. Столько же, сколько семья разнорабочего».

«У вас неточная информация. Взносы семьи среднего разнорабочего примерно 12 тысяч фунтов, поскольку у нее около 150 квадратных метров жилья, два автокара и катер. Семья доктора Туана платит меньше, потому что у них обычное жилье, но из транспорта - только мотороллер. Офиса у доктора Туана нет, он работает дома, и это его дело, не правда ли?»

Тут вмешался ведущий. Снова улыбнувшись во все 32 зуба, он спросил: «Мистер Торрес, а вы не находите несколько несправедливым, что богатый архитектор платит от заработка 0,25 процента налогов, а бедный рабочий – 30 процентов?»

«Не нахожу. В вашей стране рабочий платит 70 процентов, вот это несправедливо».

Улыбка ведущего стала еще шире. «Вы что-то путаете. У нас подоходный налог работника всего 16 процентов»

«Я ничего не путаю, - сухо возразил координатор, - я просуммировал его подоходный налог и его долю в сумме корпоративных налогов».

«Не понимаю вашей логики. При чем тут налоги с корпорации?»

«Логика элементарная. Вы знаете, что такое «прибавочная стоимость»? Корпоративные налоги выплачиваются за счет прибавочной стоимости, то есть - за счет труда работника».

«Вы – марксист?» - спросил ведущий.

«Обязательно быть марксистом, чтобы понимать, что товары производятся работниками, а не возникают по мановению волшебной палочки дирекции»? – поинтересовался Торрес.

«Вы уходите от ответа, - заявил вальяжный господин, - четверть процента и тридцать».

«Меня отвлекли. Теперь отвечу вам. Когда вы обедаете в ресторане, когда ремонтируют ваше авто, когда вы снимаете апартаменты в отеле – вы платите за обслуживание. Вам приносят счет в деньгах, а не в процентах от вашего годового дохода, не так ли?».

«Вы опять уходите…»

«Ничего подобного, - перебил координатор, - я отвечаю по существу. Вы не находите несправедливым, что обед в ресторане обойдется уборщице в такую же сумму, как вам, хотя одна булавка с вашего галстука стоит больше ее годовой зарплаты?»

В кафе раздался дружный рев одобрения – ребята поддерживали координатора так, как если бы он был боксером и отправил соперника в нокаут. Даже Викскьеф и Джой ради такого случая оторвались ненадолго от своих шашек.
- Сочинение хокку развивает ораторское мастерство, - негромко заметил Чжан.
- Вы о чем? – удивился Секар.
- Торрес пишет хорошие хокку, - пояснил китаец, - у него такое серьезное хобби.

Тем временем, в студии ведущий вновь взял слово:
«… Но, допустим, у вашего гражданина просто не хватает денег заплатить налоги… Или взносы, как вы говорите. Что тогда, мистер Торрес? Полиция перестанет его защищать?»

«Почему ему может не хватить денег?» - спросил координатор.

«Не важно. Просто не хватило - и все. Например, у него низкооплачиваемая работа».

Торрес покачал головой: «Никак невозможно. Великая Хартия запрещает условия найма, с окладом менее четырехкратного социального минимума».

«А если корпорация не может платить такой оклад?»

«Значит, ее дирекция - бестолочи. Пусть учатся управлять или ищут другое место для бизнеса. Никакое нормальное предприятие не может быть устроено так, что выручка не покрывает стоимость рабочей силы, то есть обычные жизненные потребности работников и их семей. В Меганезии хватает бизнесменов, умеющих эффективно вести дела».

«А если человек не может найти работу?»

«Тогда он идет в агентство по экономике и проходит переквалификацию. Ему будет выплачиваться стипендия. Сумму издержек по переобучению потом оплатит тот, кто примет его на работу. Это – честная сделка, от нее все выиграют».

«А если он не может работать, если он инвалид?»

«Тогда он получает страховку. Это обычная практика и у нас, и в других странах».

«Да? – с сомнением переспросил ведущий, - а откуда возьмется страховка у ребенка, который остался без родителей».

«Это – вопрос совершенно из другой области», - заметил Торрес

«Ну и что? Разве так не бывает?»

«Бывает. Уточните, вопрос о здоровых детях или о детях-инвалидах?»

«Обо всех, - сказал ведущий, - а также о детях из нищих многодетных семей».

«Тогда отвечаю по порядку. Детей, которые одновременно сироты и инвалиды очень мало и общество содержит их за счет взносов. Это мизерная сумма, никто никогда против нее не возражал. Обычные дети, оставшиеся без родителей расхватываются родственниками и семьями, у которых нет детей или есть всего один ребенок. У нас тут архаичные нравы по сравнению с Западом. Считается, что лучше, когда в семье двое или трое детей, а своих или приемных – не так уж важно. Это решет и проблему детей в нищих семьях».

«Я не понял последней фразы», - заметил ведущий.

«Это элементарно, - ответил координатор, - ребенок изымается из семьи, где его не могут достойно содержать. Далее – то же, что с детьми без родителей».

«То есть, как изымается? – выкрикнула худощавая пожилая женщина, сидящая по другую сторону стола, – по какому праву можно изъять ребенка у матери?»

«Вам хорошо известно, по какому праву, - отрезал Торрес, - судя по вашей табличке, вы представляете лигу защиты семьи. Ваша организация была депортирована из Меганезии за деятельность, несовместимую с Великой Хартией. В постановлении суда написан ответ на ваш вопрос, не так ли?»

«У вас язык не поворачивается повторить этот ответ?» - спросила она.

«Отчего же? Могу и повторить. Согласно Хартии, любой человек с момента рождения находится под защитой правительства, обеспечивающего базисные права. Если те, у кого находится малолетний, не создают условий для реализации этих прав, то малолетний передается другим лицам, готовым гарантировать его благополучие. Любые третьи лица, препятствующие этому, преследуются в порядке гуманитарной самозащиты общества».

«Самозащиты? – возмущенно переспросила женщина, - как бы не так! Это мы защищали права несчастной матери. А полицейские ворвались в ее дом, арестовали ее мужа, вырвали годовалого ребенка из ее рук. Это было бесчеловечно! Это было…»

«… Полностью правомерно, - перебил Торрес, - как и изъятие других двух детей этой женщины, возрастом два с половиной и три с половиной года, которые попрошайничали на пляже. Напоминаю мадам, что речь идет о семье сомалийцев, обитавшей в брошенном строительном вагончике, а не в доме, прошу заметить. Муж принципиально не работал, а жена не могла работать, поскольку была все время или беременной, или кормящей. Они жили мелким воровством, попрошайничеством и копанием в помойках. Оставить детей в такой семье - вот что было бы действительно…».

- О, черт! – воскликнул Лал Синг, - я знаю эту историю! Помните дискотеку на северном берегу, ну, которую держит тот парень, танзаниец с женой?
- Точно, - поддержала Инаори, - я все не могла понять: как это у них был один ребенок, а потом бац и стало четверо. Вот оно значит как…
- Эти могут хоть десяток завести, - подал голос Викскьеф, - у них по выходным половина порта куролесит. Только успевай монеты отгребать, чтоб стойка не треснула.
- Виски у него дрянь, - сообщил Эрнст, - По-моему, это вообще самогонка.
- У него так и написано: «домашнего производства», - пунктуально отметила Джой.

Викскьеф равнодушно пожал плечами:
- Виски, как виски.
Эрнст саркастически хмыкнул:
- Тебе и керосин – виски, оглобля норвежская.
Скандинав смерил его презрительным взглядом:
- Пижон. Много ты понимаешь.



10. Холодная война, пираты и каторжники.


- Ребята, дайте уже послушать! – крикнула Инаори, - там Уоррен Диксон.
- Это кто еще? – спросил Викскьеф.
- Это советник самых серьезных правительств по обе стороны Атлантики.
- Ах, вот как…

«… плохо замаскированный международный разбой, - говорил советник, - по сравнению с которым даже оффшоры выглядят безобидно. Оффшоры устраивают демпинг на рынке налогов и высасывают из развитых стран финансовые ресурсы. Но финансы все равно могут работать только в реальных экономиках, им приходится возвращаться домой. А вы сделали у себя de-facto безналоговую зону для низкоресурсных hi-tech и высасываете самые перспективные технологии производства и самых эффективных разработчиков. Ваша экономика присваивает результаты колоссальных инвестиций развитых стран в науку и образование. Вот откуда фантастический рост вашей экономики и ваше выросшее на пустом месте благополучие. Это - пиратский бизнес. Думаете, это сойдет вам с рук?»

«Как это мило, - произнес Торрес, - Когда в конце прошлого века Запад вывозил мозги из стран восточного блока, это почему-то не называлось пиратством. Ваше правительство говорило о свободе предпринимательства, экономическом соревновании и глобализации. Почему теперь эти красивые слова не звучат? Готтентотская мораль? Если я украл корову, это хорошо, а если у меня украли - это плохо?»

«То есть, вы признаете, что я прав?» - уточнил Диксон.

«Ничего подобного. Наоборот, это вы признаете, что оказались в положении Советского Союза времен холодной войны. Вы проигрываете экономическое соревнование, потому что у вас неэффективное бюрократическое управление, а частная инициатива задавлена налогами и запретами. Ваши политики сегодня только и могут, как Никита Хрущев в прошлом веке, стучать в ООН ботинком по трибуне и кричать, что они нас закопают».

Диксон усмехнулся. «Вы переоцениваете роль всяких задворков в мировой политике».

«Возможно, - сказал Торрес и задумчиво потер кончик носа, - Хотя, знаете, в конце XVI века Нидерланды казались задворками священной империи Габсбургов. Но прошло 50 лет, и Нидерланды стали процветающей республикой, владения которой раскинулись по трем океанам, а задворками оказалась как раз империя. История иногда повторяется».

«А вы уполномочены делать такие заявления? – поинтересовался Диксон, - Или хотите спровоцировать еще один международный скандал в порядке личной инициативы?»

«Заявление? – переспросил координатор, - нет, я просто напомнил кое-что из истории».

«Вы просто пытаетесь использовать это шоу, чтобы сделать рекламу своей стране».

«Конечно. Это одна из моих обязанностей, как сотрудника правительства Меганезии».

«Что ж, - сказал Диксон, - по крайней мере, здесь вы честно ответили на вопрос».

Торрес кивнул. «Честно отвечать на вопросы – это тоже моя обязанность»

Из-за стола поднялся строгий пожилой господин со значком международного бюро по правам человека: «А вы готовы честно признать, что ваше правительство игнорирует все международные гуманитарные акты?».

«Если честно – я просто не знаю всех международных актов на эту тему. Я вообще-то не юрист. Огласите весь список того, что мы, на ваш взгляд, нарушили».

«Начну со способов ведения войны. Они нарушают конвенции 1907, 1929, 1936, 1949, 1977 и 2005 года. Ваши вооруженные силы занимаются диверсиями на гражданских объектах на суше и на море, и террором против мирного населения».

«Впервые слышу о таком безобразии, - ответил Торрес, - можно конкретно?»

«Извольте. Операция ваших вооруженных сил в эмирате Эль-Шана 2 года назад. Убито 17 гражданских лиц, разрушена электростанция, центральный узел водоснабжения столицы, ВПП гражданского аэропорта и две развязки на главной национальной автомагистрали».

«Минуточку. Какие такие гражданские лица были в резиденции шейха Фархада? Если вы имеете в виду его охрану, то она была вооружена…»

«А его жена, пятеро детей, обслуживающий персонал?» - перебил представитель бюро.

Координатор пожал плечами: «Ну, знаете, это все-таки война. Наша армия, по крайней мере, не забрасывала бомбами жилые кварталы, как это принято в военной практике так называемых цивилизованных стран. Никто, кроме непосредственного окружения шейха, физически не пострадал. У жителей, конечно, были неудобства с транспортом, водой и электричеством, но в условиях войны такие вещи неизбежны».

«Но мистер Торрес, объявлять войну, физически уничтожать семьи высших чиновников государства и угрожать тотальным разрушением инфраструктуры страны из-за какого-то незначительного недоразумения с несколькими гражданами …».

«Это не было незначительное недоразумение, - отрезал он, - власти эмирата Эль-Шана захватили гражданский авиалайнер, взяли в заложники группу туристов, среди которых были наши граждане, и игнорировали наше требование вернуть им свободу».

«Но есть же дипломатические методы…»

«Есть Великая Хартия, - перебил Торрес, - Каждый гражданин Меганезии находится под безусловной защитой правительства. Эта защита не зависит ни от какой политики, ни от какой дипломатии, и осуществляется любыми средствами без всякого исключения».

«Не играйте словами! - выкрикнула дама из лиги защиты семьи, - О какой защите граждан может говорить правительство, легализующее рабовладение? Работорговля нарушает все мыслимые цивилизованные нормы!».

«Работорговля? Рабовладение? – переспросил он, - это вы о чем?»

«О вашей практике торговли каторжниками».

«Вы, вероятно, имеете в виду передачу правонарушителей в аренду предприятиям, имеющим жилые комплексы в охраняемом периметре, - сказал Торрес, - И что этим нарушается? Почему общество должно нести расходы по содержанию тюрем?»

«А как на счет надсмотрщиков с кнутами?» - спросила дама.

«На счет надсмотрщиков с кнутами вас дезинформировали», - ответил координатор.

«Но заключенных же принуждают работать, вы этого не будете отрицать?»

«Буду отрицать, потому что это – неправда. У нас принцип информированного согласия. Можно отказаться и сидеть за решеткой в одиночном боксе хоть весь срок. Но это некомфортно. Из примерно 8000 осужденных в стране это выбрали менее 50 человек».

«А каторга - это, по-вашему, комфортно?»

Торрес почесал в затылке: «Там не курорт, но ни одна международная комиссия по правам заключенных не нашла нарушений. По их данным, условия на закрытых предприятиях Меганезии примерно на уровне тюрем Швеции. Питание, быт и медицинское обеспечение практически одинаково. Оборудование спортивных площадок у них лучше, зато у нас гуманнее решен вопрос секса. У нас смешанный контингент, без ограничений на любые добровольные половые контакты. У шведов такого нет».

«Что? Как вы сказали? - Дама густо покраснела, - Вы хотите сказать, что там… на этой вашей каторге… мужчины и женщины могут друг с другом… это же гадко!»

Координатор пожал плечами: «Не понимаю, о чем вы. Суд приговорил их к лишению свободы, а не к лишению половой жизни. По мнению специалистов, принятый у нас порядок способствует исправлению и социализации правонарушителей».



11. Проблема занятий сексом в ошейнике.


- Похоже, тетка спеклась, - сказала Инаори и гаденько хихикнула.
- Да, - согласился Эрнст, - у этой карги и на свободе-то половая жизнь не сложилась.
- Страшная, как термоядерная война, - добавил Лал Синг.
- А вот эта, кстати, ничего. Очень даже, - сообщил Викскьеф

Его реплика относилась к молодой женщине, одетой в стиле «милитари», с табличкой на груди: «Жанна Ронеро. Green world press».

«В докладе бюро по правам человека было сказано, что на ваших заключенных надевают ошейники! Ошейники – на людей. Попробуйте-ка заниматься сексом в ошейнике!»

«Я бы, наверное, и вправду попробовал», - Торрес встал, развязал шейный платок и небрежно бросил его на стол. В студии послышался шум, все взгляды были обращены на шею координатора, которую охватывало полупрозрачное кольцо сантиметра 2 шириной. «Подойдите сюда, мисс Ронеро. Не бойтесь, я вас не съем».

- Зря он старается, - буркнул Лал Синг, - она ему не даст.
- Почем ты знаешь? – возразила Инаори.
- Так видно же…
- Фи! Тоже мне, Зигмунд Фрейд.

«С чего вы взяли, что я боюсь?» - вызывающе ответила Жанна, и, обойдя стол, оказалась на расстоянии вытянутой руки от координатора, «ну, я здесь, и что теперь?»

«Как видите, на мне есть ошейник слежения. Можете его осмотреть и убедиться, что он такой же, как на наших заключенных. Я надел его по требованию службы безопасности, в соответствии с пунктом моего контракта о мерах в обстановке повышенного риска».

«Подумаешь! – фыркнула она, - Вы же в любой момент можете его снять».

«Вряд ли, - возразил Торрес, - разве что я научусь отвинчивать свою голову. Чтобы снять эту штуку нужно 4 часа работы алмазной пилой».

«Или просто знать, где застежка», - язвительно дополнила Жанна.

«На нем нет застежки, мисс Ронеро, это сплошное кольцо, можете проверить»

«А я действительно проверю», - вызывающе сказала она и, протянув руку, начала ощупывать ошейник. В студии стало очень тихо. Через пару минут, журналистка достала из нарукавного кармашка пилку для ногтей и вопросительно посмотрела на координатора.

«Валяйте» - разрешил он.

Как только пилка соприкоснулась с ошейником, в кармане у Торреса зазвонил сотовый телефон. Он извлек трубку, сказал «все нормально, не беспокойтесь», и убрал назад.
Жанна, тем временем, безуспешно пыталась что-нибудь зацепить на ошейнике. При очередной попытке пилка соскользнула с гладкой поверхности и воткнулась на несколько миллиметров в шею координатора.

«О, черт!» - сказала она.

«Вы поосторожней там, - буркнул он, - я живой все-таки». В этот момент у него снова зазвонил сотовый. Он вздохнул и сказал в трубку: «ничего страшного… я же сказал, не беспокойтесь… да, под мою ответственность… хорошо, я это учту. Отбой».
По его шее стекали капельки крови, постепенно образуя на рубашке изрядное красное пятно. Фоторепортеры азартно щелкали своей аппаратурой.

«Мистер Торрес, я вызову врача и полицию», - сказал ведущий.

«Оставьте эти глупости, - недовольно бросил координатор, пристраивая на шею салфетку, взятую со стола, - подумаешь, царапина. Зато очаровательная мисс Ронеро теперь точно попадет на первые полосы. Так, мисс?»

«Извините, - пролепетала Жанна, - я нечаянно. Могу я как-нибудь…».

«Можете, - перебил он, - ужин с бутылочкой красного. Я как раз собирался попробовать местный Vineland. В рекламе пишут, что это вино сделано по древним рецептам викингов. Экзотика. Мы договорились?… Отлично. Тогда давайте продолжать».

Инаори хихикнула и повернулась к Лал Сингу.
- Ну, что, доктор Фрейд?
- Теперь другое дело, - неохотно проворчал он, - кто же знал, что она его ткнет пилкой.
- А может, она специально, - предположила Джой, - мы, женщины, коварные существа.
- Ой! – крикнула Инаори, - смотрите, дядька ну вылитый пингвин!

«… Йонсен, - сказал дядька, похожий на пингвина, - Я… э… из ассоциации морских перевозчиков. Нас интересует вопрос о безопасности судоходства, в связи с… э…»

«Пиратством», - помог ему Торрес.

«Да, верно. Именно так».

«Это проблема, - согласился координатор, - Пиратство существует в этих водах уже более 300 лет. Мы пока не полностью его уничтожили. Но наше правительство предлагает всем судовладельцам устанавливать спутниковую систему поддержки слежения. Фигурально выражаясь, это тот же ошейник, только для корабля. В случае нападения, мы гарантируем прибытие морского штурмовика за четверть часа. Любое пиратское судно уничтожается вместе с экипажем, немедленно и безоговорочно. Пиратам это известно, так что они обходят за 10 миль те суда, которые излучают на частотах нашей системы поддержки».

«Мы это знаем, сэр».

«Вот как? – удивился Торрес, - А что вам мешает заключать договоры с вооруженными силами Меганезии? Цена втрое ниже средней цены страховки груза. В чем проблема?»

«Наше правительство запрещает вступать в отношения с… - Йонсен замялся, - …с организациями вроде ваших вооруженных сил. За это у нас лишают морской лицензии».

«Ах, вот оно что. По-моему, этот запрет пахнет прямым пособничеством пиратству».

«Не знаю, сэр Торрес. Мы просто занимаемся бизнесом. Политика – не наше дело. Я хотел спросить, может быть, получится как-то уладить это дипломатически…».

«Это решается безо всякой дипломатии, - отрезал Торрес, - вам достаточно просто нанять на судно команду, состоящую в меганезийском профсоюзе моряков и не заметить, что они возьмут на борт аппарат поддержки слежения. Все. Остальное – не ваше дело».

«Что, правда? - удивился Йонсен, - а почему мы этого раньше не знали?»

«Вот это уж точно не мой вопрос. Рекомендую вам установить постоянные контакты с нашим правительственным агентством по мореплаванию. Можно - неофициальные, если официально ваши власти вам это запрещают. Мы, в отличие от некоторых правительств, занимаемся делом, а не вставляем палки в колеса бизнесу по политическим мотивам».

«Простите, - нервно заметил ведущий, - у нас время истекает…»

«Правительство установлено, чтобы обеспечить человеку пользование его естественными и неотъемлемыми правами, - продолжал Торрес, - И эти права суть равенство, свобода, безопасность, собственность. Так сказано в революционной французской конституции 1793 года, и в Великой Хартии Меганезии. Мы придерживаемся этих принципов и не отступаем от них никогда, ни при каких условиях, и ни под каким давлением».

«Да, разумеется, - еще более нервно сказал ведущий, - благодарю, мистер Торрес, что вы согласились на эту встречу в студии ABC-online, и надеюсь, что…»

Трио у телевизора разразилось хохотом.
- Обделался, - констатировал Эрнст.
- Точно, - поддержал Лал Синг.



12. Гуманитарная самозащита и хорошая реклама


… Явившись домой в седьмом часу утра, Малик, проспал до полудня. Хелена, конечно давно ушла на работу, оставив записку, прилепленную магнитом к холодильнику.

«Ты это читаешь – значит, уже открыл глазки. Надеюсь, у тебя хватит сил доползти к 5 вечера на пляж у питоновой скалы, там будет вечеринка и все такое. Люблю-целую»

Вместо подписи красовался ярко-лиловый отпечаток губ. Хелена принципиально не пользовалась губной помадой нормального человеческого цвета. Такой вот стиль…

Малик улыбнулся, почесал затылок и привычным движением ткнул две кнопки, включив кофеварку и компьютер – главные инструменты профессионального журналиста. Через минуту робот выбросил на экран свежий топ-лист заголовков статей.

В первой десятке было «интервью с правителем в ошейнике», Green world press. Малик даже присвистнул. Жанна Ронеро мастерски воспользовалась плодами своей небрежности при обращении с пилкой для ногтей. Эпизод с пилкой был представлен на фото. Дальше шел подзаголовок: «Наш бизнес - обслуживать коллективные потребности граждан по твердой смете». (Эрнандо Торрес, координатор правительства Конфедерации Меганезия).

Статья начиналась с инцидента в студии: «….Торрес демонстративно отказался от врача, и, как ни в чем не бывало, продолжал конференцию. Я подумала: он играет парня из народа. Потом, когда мы сидели в маленьком кафе и пили вино, как парочка клерков после работы, и он рассказывал, как попал в правительство, я поняла: это не игра».

Вставка: фото Торреса в кафе - видимо, снятое на сотовый телефон.

«Торрес – предприниматель в сфере туризма, - писала Жанна, - он увлеченно рассказывал о сети небольших отелей, которой он владеет вместе с компаньоном. По его словам, там он приобрел опыт, позволивший ему выиграть социальный конкурс, то есть, процедуру, по которой назначается правительство Меганезии. На вопрос, зачем он к этому стремился, Торрес ответил: это – хорошая бизнес-практика и хорошая реклама. Таков по его словам обычный мотив для участия в конкурсах на выполнение правительственных функций».

Вставка - цветная схема из квадратиков - устройство правительства Меганезии.

«Я задала Торресу 4 вопроса, которые всегда задаю политикам. Ваша самая большая неудача? Самое большое достижение? Самый смешной случай? О чем вы мечтаете?
Вот его ответы:
Неудача: я не убедил меганезийские консорциумы радикально увеличить инвестиции в фундаментальную науку. То увеличение, на которое они пошли, недостаточно. Нам необходима самая динамичная в мире фундаментальная наука, чтобы сохранить наши высокие темпы роста в ядерной энергетике, computer science и робототехнике.
Достижения: я назову два. Первое – глобальная полицейская система наблюдения. Теперь все ключевые участки территории поселений и транспортных магистралей отслеживается наземными и спутниковыми web-камерами. Видео-поток анализируется компьютером. Полиция получает сигнал тревоги через несколько секунд после возникновения любой подозрительной ситуации. Про морскую систему безопасности я уже рассказывал, она была создана предыдущим правительством. Второе – создание VECOM, общедоступной системы дистанционного высшего образования и повышения квалификации. Это дало возможность вдвое увеличить количество студентов – в первую очередь, за счет того, что молодые женщины, обзаводясь ребенком, не выпадают из образовательного процесса».

Вставка: скриншот социальной рекламы: симпатичная женщина с младенцем у монитора ноутбука. Младенец тянется пальчиками к клавиатуре. На мониторе слова: «Virtual Education Center Of Meganezia. Join now!». Подпись: «Эй, малыш, ты хочешь стать бакалавром раньше, чем научишься ходить?»

«Самый смешной случай – международное моторалли нудистов. Полиция тормознула их за езду без шлемов. Чтобы решить вопрос, я попросил департамент безопасности выдать им казенные шлемы. Ну не сообразил, что эти шлемы - форменные. Представляете?»

Вставка: скриншот репортажа CNN: по улице города едут двадцать голых раскрашенных байкеров в шлемах с эмблемой Road Police. Подпись: «Nude patrols in Meganezia».

«О чем я мечтаю? Перевернуть туристический бизнес, черт меня возьми! Как? Да очень просто! В Монреале зима, хочется на теплое море. Куда летите? Флорида. А если бы вы были в Европе? Тогда Египет. Весь фокус в цене перелета. Чуть дальше 3000 километров от дома - и перелет уже дороже отеля, да к тому же несколько часов в кресле самолета – удовольствие ниже среднего. Так вот: через два года вы полетите в Меганезию, Вам будет достаточно добраться до любого морского побережья, там катер вывезет вас на 12 миль. Дальше взлет с поверхности моря и через час вы на одном из островов Меганезии. Билет туда и обратно 200 фунтов. Отель без изысков, зато цена всего 30 фунтов в сутки и первая линия от океана - других не держим. Ах да, я забыл: по дороге вы побываете почти что в космосе. Полет в 20 раз быстрее звука на высоте более 60 километров. Это реально!».

Вставка: фото необычного летательного аппарата, подпись: Meganezia Starcraft отменяет расстояния. Европа – космос – атоллы Тихого океана за 50 минут. Подробности на сайте.

«Оказалось, это не просто смелый рекламный трюк, - писала Жанна, - Торрес рассказал историю проекта. Беспилотный перехватчик крылатых ракет был сначала переделан в пилотируемый штурмовик океанской авиации. Затем на его базе создали транспортный самолет для быстрой переброски десантников. А потом Торрес с партнером придумали переделать его в авиалайнер на 20 пассажиров. В какой-то момент я даже забыла, что разговариваю с политиком, а не с бизнесменом. Потом я поняла: в Меганезии вообще нет политической элиты. Нет даже понятия о политической карьере. Нет священной касты государственных деятелей, которую мы у себя привыкли воспринимать, как должное. Человек приходит поработать в правительство на 3 года, а потом возвращается к своему обычному делу. Для него это что-то вроде ответственной стажировки для повышения квалификации в сочетании с возможностью взглянуть на мир шире, посмотреть, чего ты стоишь, и показать умение решать действительно сложные задачи.

Привычный для нас пласт символов, связанных с государством, как со своего рода идолом, в Меганезии исчез из сознания людей. У меганезийцев есть специальное слово «оффи» - его употребляют, говоря о чиновниках или политиках какой-либо страны. Если вы спросите об отношениях Меганезии с Филиппинами - то в ответ услышите много всего о филиппинских обычаях, народной медицине и национальной кухне – но не о политике. Вы спросите об отношениях с филиппинцами – вам расскажут о конкретных этнических филиппинцах, которых много в Меганезии. Вы спросите: а как же каролинский кризис вокруг морской границы? Вам ответят: кризис не с Филиппинами и филиппинцами, а с филиппинскими оффи. Это огромная разница. В меганезийском учебнике истории сказано «1.9.1939 германские оффи послали армию в Польшу» и «20.9.1941 японские оффи послали авиацию бомбить Перл-Харбор». Вы не найдете там, что одна страна напала на другую и захватила ее, или что одна страна освободилась от колониального владычества другой. «В 1950 британские оффи утратили контроль над Индией» - так там написано. Читая этот любопытный учебник, я наткнулась на определение в рамочке:

«Государство – система открытого насилия над народом, осуществляемого в интересах олигархического клана. Олигархический клан – узкая группа людей, силой, подкупом или обманом присвоившая неограниченную политическую власть и осуществляющая ее, якобы, от имени народа. По способу присвоения власти олигархические кланы делятся на:
1. аристократические или феодальные (вооруженный захват власти)
2. плутократические или капиталистические (подкуп толпы)
3. охлократические и теократические (массовый обман)
Плутократические и охлократические режимы часто маскируются под демократические. По законам Конфедерации Меганезия покушение на создание государства, как особо опасное преступление, карается высшей мерой гуманитарной самозащиты».
 
В поисках смысла этого загадочного словосочетания я полистала учебник и нашла: «После провозглашения Великой Хартии, т.н. национальная батакская партия (НБП) совершила попытку силой захватить власть и учредить государство. Это выступление было пресечено армией Конвента. 47 лидеров НБП предстали перед верховным судом и были приговорены к высшей мере гуманитарной самозащиты - расстрелу».

После этого я поняла, из-за чего наша официозная пресса так агрессивно настроена против Меганезии. «Отцам отечества» должно быть неуютно при мысли о том, что их привычка выступать от имени нации в один прекрасный день может кончиться для них «высшей мерой гуманитарной самозащиты». Тем более, когда смотришь на нашу вцепившуюся во власть политическую элиту, такая идея не вызывает особого внутреннего протеста.

Конечно, я не так наивна, чтобы верить на слово координатору Торресу и учебнику истории. Некоторые вещи следует проверять на месте. Сейчас у меня в кармане авиабилеты, яркий цветной буклет с фотографиями атоллов и надписью Welcome to Meganezia, и памятка от правительства США: «Меганезия, как страна с экстремистским политическим режимом, признана особо неблагоприятной для туризма. Отправляясь туда, вы подвергаете свою жизнь опасности». Ну, что ж, риск – благородное дело. Читайте мой следующий репортаж – из столицы Меганезии, через три дня».

Под статьей Жанны Ронеро мигал баннер: «самые обсуждаемые темы блогосферы планеты. Меганезия – 2-е место». Малик почесал в затылке и кликнул мышкой. На экран вывалились первые 10 заголовков с анонсами.

1. Памятка – говно. Месяц назад был в Меганезии. Нет проблем ни с въездом, ни с полицией. Фестивали – отпад. Пляжи, девушки, все дела. Втрое дешевле чем Гавайи.

2. Меганезийцы - фашисты. Вас могут расстрелять прямо на улице без суда. Или продать в рабство на рудники. Запросто. И никто вам не поможет.

3. В прошлом году был на фестивале в Меганезии. Классно! Реально свободная страна! А в Вашингтоне тоже пора кое-кого прислонить к стенке. Да здравствует Великая Хартия!

4. Меганезия - та же Куба, такие же комми, только еще хуже. Они с китайцами уже поделили весь Тихий океан. Этак скоро белому человеку не будет места на планете.

5. Меганезийцы обкрадывают цивилизованный мир, вывозят наши мозги. Они хуже компьютерных пиратов. Почему Запад еще не объединился против них?

6. В Париже задолбали исламисты. Правительство их боится. Полиция ни черта не делает. Почему у нас нет такой хартии, как в Меганезии? Хватит терпеть!

7. Меганезия – сатанинская страна. Рассадник порнографии, проституции, нудизма и незаконного секса. В церквях служат черные мессы. Самое ужасное место на Земле.

8. Я из Эдинбурга. Преподаю в Меганезии высшую математику через интернет. Не знаю, что там с фашизмом, но студенты толковее, чем у нас, и за работу платят больше.

9. Эй, пипл! Через 3 дня в полдень во Франкфурте флэш-моб за меганезийскую великую хартию для Европы. 4000 человек уже вписались. Подробности у меня на сайте.

10. У нас автомастерская, половина дохода уходит на налоги. Уже достало кормить бездельников-мигрантов. Продаем жилье в Стокгольме и покупаем в Меганезии.

…

Малик хмыкнул и отхлебнул глоток уже успевшего остыть кофе. «Вот ведь жук этот Торрес! – подумал он, - на пустом месте сделал своей лавке такую рекламу за общественный счет. И ведь ни одного пункта контракта не нарушил, что характерно… Купить, что ли, акции Meganezia Starcraft? Чует мое сердце, они теперь здорово вырастут».


© Copyright: Rozoff, 2008

